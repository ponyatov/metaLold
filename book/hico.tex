% e-book
% Universal LaTeX headers for e-book publications
\documentclass[oneside,10pt]{book}
%% mobile phone optimized
\usepackage[paperwidth=118.8mm,paperheight=68.2mm,margin=2mm]{geometry}
%% font setup for screen reading
\renewcommand{\familydefault}{\sfdefault}\normalfont
%% hyperlinks pdf style
\usepackage[unicode,colorlinks=true]{hyperref}
%% fix heading styles for tiny paper
\usepackage{titlesec}
\titleformat{\chapter}{\Large\bfseries}{\thechapter.}{1em}{}
\titleformat{\section}{\large\bfseries}{\thesection.}{1em}{}
%% fix first blank page
\usepackage{atbegshi}% http://ctan.org/pkg/atbegshi
\AtBeginDocument{\AtBeginShipoutNext{\AtBeginShipoutDiscard}}
% graphics
\usepackage[pdftex]{graphicx}
\newcommand{\fig}[2]{\noindent\includegraphics[#2]{#1}}

%% bibliography
%\usepackage{titlesec}
\newcommand{\bibfig}[1]{\fig{#1}{height=.56\textheight}}

% xcolor fixes
\usepackage{xcolor}
\definecolor{red}{rgb}{0.7,0,0}
\definecolor{green}{rgb}{0,0.6,0}
\definecolor{blue}{rgb}{0,0,0.7}
\definecolor{magenta}{rgb}{0.7,0,0.7}

% Cyrillization
%% \usepackage[T1,T2A]{fontenc}
\usepackage[utf8]{inputenc}
%% \usepackage[cp1251]{inputenc}
\usepackage[english,russian]{babel}
\usepackage{indentfirst}

% relative sectioning
\usepackage{ifthen}
\newcounter{secdepth}\setcounter{secdepth}{0}
\newcommand{\secup}{\addtocounter{secdepth}{1}}
\newcommand{\secdown}{\addtocounter{secdepth}{-1}}
\newcommand{\secrel}[1]{
\ifthenelse{\equal{\value{secdepth}}{0}}{\part{#1}}{}
\ifthenelse{\equal{\value{secdepth}}{-1}}{\chapter{#1}}{}
\ifthenelse{\equal{\value{secdepth}}{-2}}{\section{#1}}{}
\ifthenelse{\equal{\value{secdepth}}{-3}}{\subsection{#1}}{}
\ifthenelse{\equal{\value{secdepth}}{-4}}{\subsubsection{#1}}{}
}
\newcommand{\secly}[1]{
\section*{#1}
\addcontentsline{toc}{section}{#1}
}
\newcommand{\subsecly}[1]{
\subsection*{#1}
\addcontentsline{toc}{subsection}{#1}
}

% misc
%% [nosep] option in lists/enums
\usepackage{enumitem}

%% typical macros
\newcommand{\email}[1]{$<$\href{mailto:#1}{#1}$>$}
\newcommand{\note}[1]{\footnote{\ #1}}
\renewcommand{\emph}[1]{\textcolor{blue}{\textbf{#1}}}
\newcommand{\cp}[1]{\note{\copyright\ #1}}
\newcommand{\term}[1]{\textcolor{green}{\textit{#1}}}
\newcommand{\termdef}[2]{\textcolor{red}{\textbf{\textit{#1}}}\index{#2}}

% comp
\newcommand{\emc}{$C$}
\newcommand{\cpp}{$C^{+_+}$}
\newcommand{\java}{$Java$}
\newcommand{\py}{$Python$}
\newcommand{\lisp}{$Lisp$}


\author{Dmitry Ponyatov \email{dponyatov@gmail.com} CC BY-NC-ND}
\title{{\Huge \ \\\hico}\\пишем язык программирования\\на Python}
\date{draft: \today}

\begin{document}

\maketitle
\tableofcontents

\clearpage
\secly{Введение}\secdown

github: \url{https://github.com/ponyatov/metaL}

\bigskip

Бесплатный черновик книги вы можете скачать на странице релизов:

\url{https://github.com/ponyatov/metaL/releases/latest}

\bigskip\noindent
Замечания и комментарии присылайте на e-mail\\или открывайте issue на github.

\subsecly{Отказ от ответственности}

Эта книга посвящена очень \emph{примитивной} реализации того, что особо
грамотные товарищи не захотят назвать языком программирования. Здесь вы не
найдете ленивых лямбд и жутких монад, живущих в лесу Хомского, и прочей
бурбулятристики про обощенный вывод типов. Тем не менее я постараюсь что-нибудь
добавить в отсутствующую нишу русскоязычной литературы по разработке языков
программирования \emph{для самых начинающих}.

\subsecly{Серебряная пуля Брукса}

Сущностью программирования является, прежде всего, не написание инструкций на
конкретном языке программирования, а выработка подробной структуры
взаимодействующих сущностей проблемной области, а также проверка внутренней
непротиворечивости этой структуры.
Следовательно, ни одно средство разработки ПО не сможет существенно снизить
сложность разработки, так как даже если, например, изобрести компьютерный язык,
оперирующий понятиями на уровне проблемной области, программирование все равно
останется сложной задачей, поскольку придется точно определять взаимосвязи между
объектами реального мира, устанавливать исключения, предусматривать все
возможные переходы между состояниями и т.д.

Что делает язык высокого уровня? Они изолируют программу от большей части ее
нецелевой сложности. Абстрактная программа состоит из концептуальных
конструкций: операций, типов данных, последовательностей и коммуникации.
Конкретная машинная программа связана с битами, регистрами, условиями, ветвями,
каналами, дисками и тому подобным. В той степени, в которой язык высокого уровня
воплощает конструкции, которые нужны в абстрактной программе, и избегает всех
нижестоящих, он устраняет целый уровень сложности, который вообще никогда не был
присущ программе. Безусловно, уровень нашего мышления о структурах данных, типах
и операциях неуклонно растет в сторону прикладной области благодаря
возможностям абстракции, предоставляемым ООП, но с постоянно уменьшающейся
скоростью\ --- развитие языков и абстракий движется все ближе и ближе к
прикладной сложности пользователей. Более того, в какой-то момент разработка
языка высокого уровня и фреймворков создает бремя владения инструментом, которое
увеличивает, а не уменьшает интеллектуальную сложность пользователя.

Многие люди ожидают, что достижения в области искусственного интеллекта
обеспечат революционный прорыв, который даст увеличение производительности и
качества программного обеспечения на порядок. При этом не стоит путать
"численный ИИ" как он широко известен сейчас, т.е. нейроные сети и машинное
обучение, с \term{семантическим ИИ}, выполняющим логический вывод \emph{на
основе сетей взаимосвязанных понятий и отношений между объектами}. Наиболее
широко известная технология семантического ИИ\ --- \term{экспертные системы}.

\begin{quotation}\noindent
Экспертная система\ --- это программа, которая содержит обобщенный механизм
логического вывода и базу правил и отношений, принимает входные данные и
предположения, генерурет гипотезы, выводимые из базы правил, дает выводы и
рекомендации, и предлагает объяснение свои результатов для пользователя (путем
отслеживания цепочки логического вывода). Механизмы вывода часто могут иметь
дело с нечеткими или вероятностными данными и правилами, в дополнение к чисто
детерминированной логике.
\end{quotation}

Как эта технология может быть применена к задаче разработки программного
обеспечения?

Работа, необходимая для генерации базы знаний\ --- это работа, которую в любом
случае необходимо будет не просто выполнить единожды, но и постоянно
поддерживать базу знаний в актуальном состоянии. Многие трудности стоят на пути
скорейшей реализации полезных экспертных системных советников для разработчика
программ. Важной частью нашего воображаемого сценария является разработка
простых способов перехода от спецификации структуры программы к автоматической
или полуавтоматической генерации кода, созданию правил тестирования и
диагностики, скриптов развертывания и средств мониторинга. Еще более трудной и
важной проблемой является получение знаний в двух направлениях: поиск четких,
самоаналитических экспертов, которые знают, почему они делают что-то, и
разработка эффективных методов извлечения того, что они знают, и их
использование в базах правил. Необходимым условием для построения экспертной
системы является наличие эксперта.

Самым мощным вкладом экспертных систем, безусловно, должно быть предоставление
на службу неопытному программисту опыта и накопленной мудрости лучших
программистов. Это немалый вклад. Разрыв между лучшей практикой разработки
программного обеспечения и средней практикой очень велик\ --- возможно, больше,
чем в любой другой инженерной дисциплине. Инструмент, который распространяет
передовой опыт, был бы важен.


\subsecly{Языково-ориентированное программирование}

Языково-ориентированное программирование\ --- разработка, опирающаяся на
предметно-специфичный язык (англ. DSL\ --- Domain-Specific Lan\-guage).
Это парадигма программирования, заключающаяся в разбиении процесса разработки
программного обеспечения на стадии
\begin{itemize}[nosep]
  \item 
разработки предметно-ориентированных языков (DSL) и
  \item 
описания собственно решения задачи с их использованием.
\end{itemize}


\clearpage
\subsecly{Метапрограммирование}\label{meta}

\begin{quotation}\noindent
Метапрограммирование — вид программирования, связанный с созданием
\textit{программ, которые порождают другие программы} как результат своей работы
(в частности, на стадии компиляции их исходного кода), либо программ, которые
меняют себя во время выполнения (самомодифицирующийся код).
\end{quotation}

\begin{itemize}
  \item 
\url{https://www.youtube.com/watch?v=QKFrxEkVusg}
  \item 
\url{https://www.youtube.com/watch?v=bt6kU1kuHWA}
\end{itemize}

\noindent
Традиционно при написании программ стараются писать код максимально переносимым
между различными компиляторами. ОС и аппаратурой, для этого создают различные
фреймворки, HAL, стандартные библиотеки и т.п. В итоге вместо быстрых
эффективных программ получаются \textbf{Jаба}троны завернутые в десятки слоев
абстракций и выжирающих ОЗУ гигабайтами\note{Eclipse на запуске на пару минут
вырубает не самый тухлый i7}.
Метапрограммирование через генерацию кода способно решить обратную задачу:
получение исходного кода на \textit{embedded \emc}\ \note{\cpp, \java\ или любом
другом языке, в т.ч. на \py\ для самораскрутки системы}\ максимально
учитывающего все особенности используемой аппаратуры, окружения и конкретной
решаемой задачи. Общая идея\ ---
\begin{itemize}[nosep]
  \item 
\emph{шаблонизация}, 
  \item 
\emph{параметризация} и 
  \item 
\emph{наследование} \textbf{исходного кода}
\end{itemize}
написанного на языках программирования, которые в принипе не знают об ООП,
наследовании и шаблонах (ISO \emc, Makefile, МЭК 61131-3), или не способных их
полноценно реализовать\ \note{интерересно через сколько десятилетий наконец
додумаются встроить в компилятор \cpp\ интерпретатор (\lisp а?) для построения
кода в compile time, вместо сомнительных шаблонов?}. Вся абстрактная каша должна
оставаться на рабочей станции разработчика в высокоуровневом \py-коде,
результат\ --- низкоуровневый код на \emc/LLVM способный работать на сотнях байт
ОЗУ\note{типичное требование для прошивок аппаратуры, сделанных на дешевых
low-end микроконтроллерах, имеюших всего \emph{2+ Кило}байта ОЗУ}.

Идеальным результатом применения metapy будет код, не выполняющий ни одной
машинной инструкции, которая не является необходимой для инициализации
конкретной железки, или решения текущей задачи. Если код должен работать поверх
ОС, в идеале он должен использовать только нативный API и
\term{специфицированный} код
вместо сторонних библиотек и особенно мультиплатформенных фрейморков. В
реальности естественно приходится ограничиваться точечным применением, т.к. есть
legacy код, требования к читаемости выходного кода, обучение программистов
сложной методике, сложность реализации вывода (компиляция мета-моделей), и
невозможность переписать в виде метамоделей весь используемый набор сервисов и
библиотек.

\subsecly{Гомоиконичные языки программирования}\label{homoiconic}

\secup


\part{Обзор и применение \hico}

Интро еще толком не прописал, поэтому вкратце:
интересует применение экспертных систем для генерации программ для встраиваемых
систем и IoT,

погуглил на тему представления знаний в таких системах, попалась
книжка Марвина Минского (в переводе \cite{minsky}) и пара ссылок с кратким
описанием принципа, подкупает полная поддержка ООП и объектного представления + логический вывод

\begin{itemize}[nosep]
  \item 
прототип решил делать поверх Python\note{чтобы не возиться с управлением
памятью, и использовать несколько удобных библиотек},\\
  \item 
командный язык а-ля Форт (стек и постфикс) для простоты,\\
  \item 
как основной инструмент хочется унификацию (как в Прологе) но более
дружественную к императивному программированию,\\
\end{itemize}
и самое главное гомоиконичность\\
(а) чтобы система могла достраивать сама себя (bootstrap) и\\ 
(б) \emph{полностью динамическая интерактивная система а-ля Smalltalk/Self}\\
позволяющая в себя залезть/отладить/модифицировать в рантайме

\bigskip
\noindent
основное прикладное применение: \emph{генерация кода для микроконтроллеров по
шаблонам}\ \note{параметрические куски кода на embedded Си, которые немного
изменяются в зависимости от целевой системы и контекста в котором используются}
наследование дизайна прошивки: есть код прошивки для базового прибора, и
полсотни заказчиков, каждый хочет свои лыжи и гамак, С++ под корпоративным
запретом (и в 2-8К ОЗУ не разбежишься), в итоге исходники неконтролиремо
копипастятся и имеют море наслоений legacy


\part{Реализация в деталях}\secdown

\secrel{Концепция фреймов Марвина Мински}\label{frame}\secdown

\clearpage
\cite{minsky} Марвин Минский \textbf{Фреймы для представления знаний}

\begin{itemize}
%   \item 
% \url{https://royallib.com/read/minskiy_marvin/freymi_dlya_predstavleniya_znaniy.html#0}
  \item 
\url{https://ponyatov.quora.com/Minsky-Frames-Database-metaL}\\(см. видео в
начале)
\end{itemize}

В качестве модели представления (мета)программ было выбрано расширенное
представление фреймов Мински. Оригинальные фреймы не имели очень важного для
метапрограммирования функционала: \textit{способности хранить упорядоченные
объекты}. Эта фича необходима для представления любых
программ\note{последовательного набора инструкций, или рекурсивно вложенных
структур}, в качестве примера см. деревья разбора/AST и реализацию атрибутных
грамматик \cite{dragon2}. С другой стороны, фреймы имеют практически полное
соответстивие объектной парадигме, в т.ч. объектам \py.

Если мы попытаемся описать дерево программы через граф объектов (фреймов), мы
сталкиваемся с необходимостью иметь \emph{упорядоченные контейнеры}, например
для хранения операндов в выражении деления. Одновременно нам необходим
\emph{ассоциативный массив} для хранения и обработки \term{атрибутов}\ при
преобразованиях кода с использованием \term{атрибутных грамматик}.

Оригинальная модель фреймов не предусматривает упорядоченное хранение, поэтому
был выбран расширенный вариант модели, для некоторой универсализации,
\begin{itemize}
  \item 
выделенная иерархия классов применяется для отделения логики фреймов от логики
работы объектной системы в Python\ \note{хотя в принципе динамическая природа
\py\ позволяет реализовать все на встроенных механизмах его объектного движка},
  \item 
явные манипуляции с фреймовыми структурами демонстируют принципы реализации на
низкоуровневых языках с жесткой типизацией, AOT-компиляцией и соответственно
невозможностью произвольно менять структуру класса или единичного объекта в
рантайме (\cpp, \java)
  \item 
добавление некоторых фич, характерных для функциональных и логических языков 
программирования \note{унификация/backtracking и структурный pattern matching}
дает возможности, крайне полезные для метапрограммирования и реализации
интеллектуальных систем (базы знаний, экспертные системы, \term{семантический
ИИ}).
\end{itemize}

\lst{lst/frame.py}{language=Python}
 
% , предложенного Марвином Мински,
% добавлением функционала упорядоченного контейнера `nest[]`, позволяющего
% не только хранить `attr{}`ибуты (слоты),
% но и любые элементы данных в явно заданном порядке.
% 

% 
% Также в большинстве случаев у нас есть необходимость хранить для любого
% элемента данных два поля:
% * `type` <br>
% явно указывающий на тип фрейма. Мы принципиально не можем оперировать
% двумя фреймами в выражении типа `<string:> + <number:>` без их приведения к
% одному типу, причем это приведение часто зависит от контекста, в каком именно
% смысле мы это выражение используем (привет долбанутый JavaScript)
% * `value` <br>
% атомарное значение, хранимое в типе языка реализации (Python): нам нужно
% именовать объекты, хранить значение строк и числовых данных, поэтому также
% необходимо подкласс фреймов для представления таких значений-примитивов.
%   
% (*) имена type/value фиксированы требованиями библиотеки PLY, если вы захотите
% использовать ее для создания собтвенного языка метапрограммирования или CLI
% вместо Python

\secup

\clearpage
\secrel{Язык \metal: исправленный \F}\secdown

Хотя мы стараемся уйти от использования языка программирования как основного
средства разработки \ref{nolang}, в любом случае нам нужен способ ввода данных и
систему, и управления вычислениями.

\emph{\metal\ не является языком
программирования}, это \term{командный язык} с помощью которого выполняется
\begin{itemize}[nosep]
  \item 
создание фреймов, 
  \item 
модификация \term{фреймовой базы знаний}, 
  \item 
запуск/останов скриптов и демонов. 
\end{itemize}
Однако очень простое \term{императивное
программирование} может выполняться и на \metal, так как этот язык позволяет
определять новые \F-\term{слова}, и поддерживает \term{конкатенативное
программирование} через разделяемый стек.

\clearpage
В качестве прототипа для \metal\ был выбран язык \emph{\F: это самый
элементарный язык программирования}, который вы только можете найти. Вы можете
самостоятельно написать свой \F\ за пару вечеров или пару недель на любом языке
программирования, и для любого типа компьютера.

% \smallskip\noindent
\F\ был создан в 70х годах Чаком Муром для управления оборудованием
(радиотелескопом), и \emph{\F\ до сих пор великолепен в роли командной оболочки}
(CLI) для подобных задач. В том числе \F\ очень хорошо подходит как командная
консоль для микроконтроллеров с очень небольшими объемами ОЗУ порядка 8-20
Кило(!)байт.

Но в роли основного языка программирования \F\ очень плох:
\begin{description}[nosep]
\item[низкоуровневая модель ВМ языка]: \F\ по факту является ассемблером
\term{виртуальной стековой машины}, и как с любым ассемблером вам приходится
самостоятельно выписывать все фишки, которые в mainstream языках доступны из
коробки в базовой спецификации языка. Как пример, в стандартное ядро языка не
входит поддержка вычислений с плавающей точкой, и полностью отсутствуют средства
динамического выделения памяти.
\item[прямой доступ к памяти по адресам]\ делает код на \F е крайне
нестабильным: ошибки в адресации тут же приводят к перезаписи данных и кода по
случайным адресам. В результате при программировании на \F\ нужно работать с
памятью на порядок аккуратнее, чем на \emc.
\end{description}

\secup

\secrel{Web-интерфейс /Flask/}\label{web}\secdown
\secup

\secrel{Элементы языка \prolog}\secdown

\begin{itemize}[nosep]
  \item 
\url{http://yieldprolog.sourceforge.net/tutorial1.html}
  \item 
\url{http://yieldprolog.sourceforge.net/tutorial2.html}
\end{itemize}

\secrel{Магия алгоритма унификации}\secdown

\secrel{Генераторные функции и yield}\label{yield}

Ключевое слово \file{yield}\ в \py\ превращает любую функцию, в которой оно
используется, в функцию-\term{генератор}. Вызов генератора вместо
выполнения функции возвращает объект-\term{итератор}. Если его использовать в
качестве параметра цикла \file{for}, или явно вызывать встроенй метод
\verb|__next__()|, то вы сможете использовать \term{ленивые вычисления}\ в
обычной императивной программе на \py.

\begin{quotation}\noindent
\term{Ленивые вычисления} (англ. lazy evaluation, также отложенные вычисления)\
--- применяемая в некоторых (функциональных) языках программирования стратегия
вычисления, согласно которой вычисления следует откладывать до тех пор, пока не
понадобится их результат.
\end{quotation}

В рамках \py\ полная реализация ленивых вычислений недоступна \ref{lazy}, тем не
менее использование генераторов позволяет вычислять функции в бесконечном цикле,
возвращая промежуточные результаты. Также на генераторных функциях построен
механизм \term{логического вывода в возвратами}, используемый в языке \prolog,
который мы рассмотрим далее.

\medskip
\lst{prolog/00.py}{language=Python}

Генератор \file{person()}\ соответствует 0-арному \term{отношению}
\file{person()}, которое определяет свойство быть человеком (person) для
некоторых внешних объектов, которые явно не указаны в качестве параметров
отношения.

\begin{quotation}\noindent
\term{отношение}\ --- свойство некоторого объекта, или связность нескольких
объектов между собой.
\end{quotation}

\begin{quotation}\noindent
\term{арность}\ --- число объектов: параметров отношения
\end{quotation}

\begin{quotation}\noindent
\term{предик\'{а}т} (n-местный, или \term{n-арный})\ --- это логическая функция
с множеством значений \{0,1\} или \{false,true\} (\{ложь, истина\}),
определённая на множестве $M=M_1 \times M_2 \times \ldots \times M_n$. Таким
образом, каждый набор элементов множества $M$ характеризуется либо как
``истинный'', либо как ``ложный''.
\end{quotation}
Предикат можно связать с математическим \term{отношением}: если кортеж\\
$(m_1,m_2,\dots ,m_n)$ принадлежит отношению, то предикат будет возвращать на
нем \file{true}. В частности, одноместный предикат определяет отношение
принадлежности некоторому множеству.

\begin{description}[nosep]
\item[унарное отношение] \file{relation(object)}\\
определяет некоторое свойство объекта, задает \term{множество} объектов,
обладающих этим свойством, и соответствует одноименной функцию-предикату
способную проверить обладает ли \file{object} заданным свойством \file{relation}
\item[бинарное отношение] \file{binar(obj1,obj2)}\\
связывает два объекта
\item[n-арное отношение] \file{Nary(obj1,\ldots)}\\
\verb|sum(A,B,product)| задает \term{тернарное} отношение суммы: \verb|product|
является суммой \verb|A| и \verb|B| (порядок параметров предиката важен, но не
предопределен)
\item[нуль-арное отношение] \file{time()}\\
обобщение, отношение заданное для внешних, явно не указанных, неименованных
объектов; например текущее время, или состояние системы. Такие объкты можно
задавать и логической переменной, но описание отношения принадлежности строк
файлу \verb|FileName()| получится значительно сложнее, многословнее, и скорее
всего с использованием рекурсии.
\end{description}

\bigskip
Генератор (отношение) может быть задан и для бесконечного множества значений,
например бесконечной последовательности, и как раз здесь срабатывает принцип
ленивых вычислений: каждое новое значение вычисляется по необходимости, в нужный
момент, при этом не требуется\note{потенциально бесконечная, или
непредсказуемая} резервация памяти для хранения данных.

\medskip
\lst{prolog/01.py}{language=Python}
\lst{prolog/02.py}{language=Python}

\secrel{Логические переменные}

Переменные в языке \prolog\ используются в процессе \term{логического вывода}
как средство передачи значений между многими предикатами. Попробуем аналогично
использовать переменную \py:

\medskip
\lst{prolog/var03.py}{language=Python}

То же самое в системе фреймов: заводим новый класс переменной:

\lst{prolog/var04.py}{language=Python}

Генераторная функция \term{связывает переменную} со значением, и
\emph{возвращает переменную} как результат на каждой итерации:

\lst{prolog/var04.out}{}

\secup
\secup
\secrel{Динамическая компиляция}\label{dyna}\secdown

\secrel{LLVM}\label{llvm}\secdown
\url{https://www.youtube.com/watch?v=q6uF3a-SJUU}
\secup


\secup

\secup

\addcontentsline{toc}{section}{Литература}
\begin{thebibliography}{99}

\clearpage
\bibitem{py}\ \bibfig{bib/python.jpg}\\
\textbf{Язык программирования Python}\\
Россум, Г., Дрейк, Ф.Л.Дж., Откидач, Д.С.,\ldots\\
\url{http://rus-linux.net/MyLDP/BOOKS/python.pdf}

\clearpage
\bibitem{dragon2} \bibfig{bib/dragon2.png}\ \emph{Purple Dragon Book /2nd ed/}\\
\textbf{Компиляторы: принципы, технологии и инструментарий} 2 изд.\\
Альфред В. Ахо, Моника С. Лам, Рави Сети, Джеффри Д. Ульман.\\
М.: Вильямс, 2008.\\ ISBN 978-5-8459-1349-4.\\
\url{https://www.ozon.ru/context/detail/id/148568229/}

\clearpage
\bibitem{sicp} \bibfig{bib/sicp.jpg}\ \textbf{\emph{SICP}\\
\href{https://drive.google.com/file/d/0B0u4WeMjO894X3lnWmhjUktKRk0/view?usp=sharing}{Структура
и интерпретация компьютерных программ}}\\
Харольд Абельсон, Джеральд Сассман\\
ISBN 5-98227-191-8\\
EN: \url{web.mit.edu/alexmv/6.037/sicp.pdf}\\
\url{https://www.ozon.ru/context/detail/id/5322055/}

\clearpage
\bibitem{bratko}\ \bibfig{bib/bratko.jpg}\\
\textbf{Программирование на языке Пролог\\для искусственного интеллекта}\\
Иван Братко\\
Мир, 1990\\ ISBN 5-03-001425-Х, 0-201-14224-4

\clearpage
\bibitem{minsky}\ \bibfig{bib/minsky.jpg}\\
\textbf{Фреймы для представления знаний}\\
Марвин Минский\\
 М.: Мир, 1979.\\
\url{https://royallib.com/book/minskiy_marvin/freymi_dlya_predstavleniya_znaniy.html}\\
\url{https://www.ozon.ru/context/detail/id/31747338/}

\clearpage
\bibitem{starting}\ \bibfig{bib/starting.jpg}\\
\textbf{Начальный курс программирования на языке ФОРТ}\\
Лео Броуди \\
М.: Финансы и статистика, 1990. - 352 с.  ISBN 5-279-00262-6.\\
Пер. с англ. В.А.Кондратенко. Под ред. Б.А.Кацева, В.А.Кириллина. Предисловие
И.В.Романовского.\\
\url{http://www.nncron.ru/download/sf.pdf}

\end{thebibliography}


\end{document}
