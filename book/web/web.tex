\secrel{Web-интерфейс /Flask/}\label{web}\secdown

Система по умолчанию запускается в консольном режиме, и не требует никаких
сторонних библиотек. Это удобно при запуске \hico\ в консольном окне
\eclipse, или в командной строке. Если вы хотите удобства, например иметь
несколько окон, показывающих состояние различных объектов в словаре системы,
стоит перейти к Web-интерфейсу, обслуживаемому \term{web-фрейморк}ом Flask. Он
не так известен как Django, это минималистичен и не тянет за собой массу
библиотек, базу данных, и не требует для работы настройку виртуального
окружения.

\medskip
\lst{web/web.py}{language=Python}
\begin{description}[nosep]
\item[\file{WEB()}] весь веб-интерфейс завернут в функцию\\ если у вас не
установлены библиотеки Flask, до попытки ее запуска никаких проблем не
возникнет, консольный режим будет работать
\item[\file{IP}] сервис запускается на \file{localhost}.\\ \emph{не запускайте
\hico\ на открытых IP}, так как веб-интерфейс запускается в отладочном режиме, и
становится дыркой в безопасности.
\item[\file{PORT}] порт сервиса\\ при запуске от пользователя в \linux\ доступны
порты выше 8000
\item[\file{web}] Flask-приложение
\item[\file{SECRET\_KEY}] используется в защите веб-форм от cross-site атак\\
инициализируется примитивно\ --- случайной строкой
\item[\file{.route}] роутинг задает маршрутизацию веб-запросов\\ привязывает
путь к ресурсу в URL к прикладному коду, оформленному как функция-обработчик
запроса
\item[\file{index()}] обработчик \file{http://127.0.0.1:8888 /}\\
в простейшем случае нам нужно обрабатывать запросы только к корневому URL, и
отдать кусок html, в примере отдается дамп словаря в plain text завернутый в тег
\verb|<pre>|
\item[.run(ip,port,debug=True)] запуск веб-сервиса в отладочном режиме\\ при
появлении исключений при запуске ваших команд будет выведен дамп ошибки, и
сервис не прекратит работу аварийно
\end{description}

\smallskip
\fig{web/first.png}{height=.5\textheight}

\secup
