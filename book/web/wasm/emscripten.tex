\secrel{\ems: старт на \cpp}\secdown

\noindent
\url{https://tproger.ru/translations/introduction-to-webassembly/}

\bigskip
Начинать проще с высокоуровневого тулчейна для \emc/\cpp\ --- для него проще
найти tutorialы, и собрать необходимые библиотеки. 

\secrel{Установка}

\noindent
SDK \ems\ предоставляет обширный набор инструментальных средств для разработки
на \cpp\ для фронтенда. \ems\ не поставляется в виде готовых пакетов для Debian
\linux, и использует собственную систему установки со встроенной поддержкой
обновлений до новых версий SDK:
\begin{lstlisting}
$ cd ~
$ git clone --depth 1 \
	https://github.com/emscripten-core/emsdk.git
$ sudo apt install git cmake nodejs python2.7
$ cd emsdk

~/emsdk$ git pull
~/emsdk$ ./emsdk update
~/emsdk$ ./emsdk install latest
~/emsdk$ ./emsdk activate latest
~/emsdk$ source ./emsdk_env.sh
\end{lstlisting}

\secrel{Первые программы}

\noindent
\url{https://tproger.ru/translations/webassembly-tutorial-first-steps/}

\bigskip
\lst{web/wasm/none.c}{language=C}
\lst{web/wasm/Makefile}{}
На выходе получаем бинарный файл, содержащий 
\begin{lstlisting}[title=none.wasm]
0000000 060400 066563 000001 000000 007400 062006 066171
0000020 160153 140201 002002 000000 000400 002426 000540
0000040 000540 000577 060177 000400 060177 077402 000177
0000060 001000 000654 001411 067145 011166 061141 071157
0000100 060564 065543 073117 071145 066146 073557 000000
...
\end{lstlisting}

\secup
