\secrel{WasmFiddle}

\noindent
\url{https://wasdk.github.io/WasmFiddle/}\\
\url{https://habr.com/ru/post/342180/}
\bigskip

Попробовать технологию без установки можно online.

\bigskip
\lst{web/wasm/none.c}{language=C}
% \bigskip

Слева вверху исходный код, слева внизу текстовое представление результата
компиляции по кнопке Build, справа вверху \js\ код для запуска и справа
внизу результат запуска по кнопке Run.

WebAssembly это бинарный формат, для удобства и отладки существует механизм
текстового представления один-в-один в виде текста в формате WAT, по синтаксису
аналогичен описанию вложенных списков (дерево программы) на языке \lisp.

\lst{web/wasm/none.wat}{title=none.wat}
\begin{description}
\item[\file{(memory \$0 1)}] памяти выделяется страницами по 64К
\item[\file{(export "none" (func \$none))}] функция экспортируется в
\js-окружение
\end{description}

% \clearpage
Загрузка модуля выполняется скриптом в .html

\lst{web/wasm/none.html}{language=html,title=none.html}
Поскольку в локальном варианте загрузка .wasm файла недоступна, байт-код
приходится вставлять непосредственно в скрипт, используя режим вывода Code
Buffer в интерфейсе WasmFiddle.

Загруженный модуль экспортирует память и функцию, скомпилированную при загрузке
в машинный код процессора в память процесса браузера:\\
\fig{web/wasm/none.png}{height=.65\textheight}

\lst{web/wasm/hello.c}{language=C}
\lst{web/wasm/hello.wat}{title=hello.wat}
\lst{web/wasm/hello.html}{language=html}
