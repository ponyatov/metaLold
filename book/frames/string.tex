\secrel{Строка}\label{string}

Строка\ --- самый универсальный тип данных, способный представить любой другой.

\lst{lst/string.py}{language=Python}

\noindent
Как было упомянуто в \ref{dumpstr}, для текстового представления значений
объектов часто нужна специальная обработка. В случае строки, здесь мы
переопределяем метод \verb|str()| чтобы получить однострочную запись для строк,
содержажих служебные символы-разделители, т.е. многострочные тексты.
Поддерживается несколько видов служебных символов, базовыми являются табуляция
\verb|'\t'| и конец строки UNIX \verb|'\n'| посколько они входят в вывод дампа и
в большинство файлов с исходным кодом (как первичные примеры многострочного
текста).

Необходимость такого преобразования вы поймете как только попытаетесь работать с
чем-то типа синтеза исходного кода или генерацией .html. Когда мы читаем дамп,
кроме значений элементов нам не менее важна их структура, прежде всего уровень
вложенности. Как только в дамп попадает многострочная строка, и так плохо
читаемый дамп легким движением руки превращается в кашу, способную легко уделать
программу на \lisp е.
