\secrel{Дамп фрейма}

Для работы с фреймами нам необходимо их человеко-читаемое представление. В
разделе \ref{graphdump}\ описано более естественное отображение в виде графа, но
даже на начальном этапе нам нужен способ видеть их внутреннее состояние.
Наиболее информативным и простым в реализации является представление фрейма в
виде текстового дерева. Каждый фрейм выводится в виде \emph{короткой формы}
\verb|<тип:значение>|, в \emph{полной форме} за ней следует табулированное
дерево, отображающее другие вложенные фреймы.

\smallskip
\lst{lst/dump.py}{language=Python}

\begin{description}
\item[\_\_repr\_\_()] переопределение системного оператора для вывода
\verb|print|
\item[dump()]\ полнотекстовый вывод фрейма в виде текстового дерева
\begin{description}
\item[depth]\ текущая глубина вложенности\\
обход фреймов выполняется рекурсивно, на каждом шаге рекурсии вызывается метод
\verb|dump(depth+1)|, в результате чем глубже вложен фрейм, тем с большим
отступом он будет выведен
\item[prefix]\ ghtabrc при выводе слотов (атрибутов)\\
в начало заголовка фрейма добавляется префикс, содержащий имя слота:
\verb|slotname = <type:value>...|
\end{description}
\end{description}

\smallskip
\lst{lst/dumphead.py}{language=Python}
 \begin{description}
\item[head()]\ дамп в короткой форме\\
выводится только тип и значение фрейма в короткой строке,\\
в некоторых случаях необходимо отличать \textit{копии} фреймов или
\textit{разные} фреймы с одним типом и именем, поэтому через @ выводится
\verb|id()| объекта в шестнадцатеричном виде
\end{description}

\lst{lst/dumpadstr.py}{language=Python}

\begin{description}
\item[str()]\ в простейшем случае кажется естественным напрямую
использовать \verb|Frame.value|, но в некоторых случаях необходимо обеспечить
конкретную форму вывода значения, как пример целые числа или hex; поэтому
введен \emph{переопределяемый метод} для получения value.
\end{description}

\paragraph{Бесконечный рекурсивный цикл при выводе дампа}\ \\

Фреймы могут включать самих себя в качестве значения или вложенного элемента
(непосредственно или в составе других фреймов). Попытка вывода дампа таких
вложений приводит к бесконечной рекурсии и аварийному завершению интерпретатора.

Для борьбы с этим эффектов в класс \verb|Frame| введен статический список
\verb|dumped[]| хранящий указатели на все фреймы, которые уже были выведены.
Это список обнуляется при первом выводе корневого фрейма (\verb|depth=0|), а
затем каждый раз проверяется, есть ли уже текущий объект в этом списке: если
есть\ --- возвращается только короткая форма с \ldots, иначе фрейм добавляется.

\clearpage
\lst{lst/dumped.py}{language=Python}
