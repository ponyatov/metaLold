\secrel{Примитивные типы}\label{prim}\secdown

Примитивный (встроенный, базовый) тип\ --- тип данных, предоставляемый языком
программирования как базовая встроенная единица языка. Обычно о примитивных
типах говорят с точки зрения компилятора, предоставляющего набор типов,
реализующихся на машинном уроне.

В нашем случае фреймы обеспечивают представление верхнего уровня для таких
типов, но при этом предоставляют тот же набор интерфейсов, что и остальные типы:
\emph{в \hico\ примитивные типы} (число или строка) \emph{могут иметь
произвольные вложенные элементы и атрибуты, и поддерживают общие для всех
фреймов операции}.

Такой подход перевешивает по уровню абстракции даже подход \py\ ``все есть
объект'', и значительно замедляет прямые попытки делать на \hico\ какие-либо
вычисления. На самом деле это \textit{кривые} попытки\ --- если вы это делаете,
значин сами себе злобные Буратины, для этого предназначены другие методы
\ref{dyna}. Не нужно забывать что \emph{\hico\ это язык для
метапрограммирования} и приложений искуственного интеллекта для этого самого.

Например если вы захотите написать САПР и использовать числа как единицы
изменений размеров, вам придется создавать целую пачку специализированных
классов для представления единиц, назначения допусков, используя примитивные
типы как безразмерные величины, к которым искуственно добавляются дополнительные
атрибуты. У \hico\ несколько иной подход\ --- число может представлять любую
величину, достаточно добавить к нему атрибуты или указать единицы измерения как
вложенный элемент (по вашему выбору).

По факту, в \hico\ базовыми типами являются типы языка реализации (\py), Тем не
менее в случае (динамической) компиляции примитивные типы будут преобразованы в
машинные, если набор атрибутов не заставит процесс генерации кода использовать
дополнительные обертки.

\secrel{Строка}\label{string}

Строка\ --- самый универсальный тип данных, способный представить любой другой.

\lst{lst/string.py}{language=Python}

\noindent
Как было упомянуто в \ref{dumpstr}, для текстового представления значений
объектов часто нужна специальная обработка. В случае строки, здесь мы
переопределяем метод \verb|str()| чтобы получить однострочную запись для строк,
содержажих служебные символы-разделители, т.е. многострочные тексты.
Поддерживается несколько видов служебных символов, базовыми являются табуляция
\verb|'\t'| и конец строки UNIX \verb|'\n'| посколько они входят в вывод дампа и
в большинство файлов с исходным кодом (как первичные примеры многострочного
текста).

Необходимость такого преобразования вы поймете как только попытаетесь работать с
чем-то типа синтеза исходного кода или генерацией .html. Когда мы читаем дамп,
кроме значений элементов нам не менее важна их структура, прежде всего уровень
вложенности. Как только в дамп попадает многострочная строка, и так плохо
читаемый дамп легким движением руки превращается в кашу, способную легко уделать
программу на \lisp е.

\secrel{Cимвол}\label{symbol}

\lst{frames/symbol.py}{language=Python}

\secrel{Числа}\label{number}\secdown

\lst{lst/number.py}{language=Python}

\secup


\secup
