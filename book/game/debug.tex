\secrel{Отладка под эмулятором}\label{qemudebug}

QEMU обеспечивает не только запуск \term{bare metal} программ в эмулиремой
виртуальной машине, но и полноценный отладочный интерфейс для GNU \file{gdb}.

\lst{game/debug0.mk}{title=\file{/game/Makefile},language=make}

PHONY-цель \file{go}\ идет в \file{Makefile} самой первой, и является главной
целью при запуске \verb|make [go]|. \file{game.elf}\ будет автоматически
перекомпилирован если вы меняли или удаляли какие-то зависимые файлы (с
исходным кодом), затем будет запущен QEMU в режиме \term{отладочного сервера}.
Режимы работы задаются с помощью опций командной строки:

\begin{description}
\item[-kernel game.elf] QEMU умеет загружать напрямую ядра операционных систем и
bare metal программы, в которых есть загрузочный заголовок Multiboot
\ref{multiboot}. Для запуска игр скомпилированных в ELF файлы вам не нужны
образы дисков\ --- все ресурсы игры вкомпилированы внутрь исполняемого
кода, и загрузчик ОС сам позаботится чтобы они были помещены в правильные
области ОЗУ.
\item[-nographic] отключить открытие окна эмулятора видеовывода VGA
\item[-s] включить \term{gdb-сервер}
\item[-S] ожидать подключения отладчика до запуска программы
\end{description}

\begin{framed}\noindent
Умение пользоваться \term{отладчиком}\ --- \emph{критический навык} для
разработчиков встраиваемых систем.
\end{framed}
