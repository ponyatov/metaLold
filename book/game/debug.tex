\secrel{Отладка под эмулятором}\label{qemudebug}\secdown

QEMU обеспечивает не только запуск \term{bare metal} программ в эмулиремой
виртуальной машине, но и полноценный отладочный интерфейс для GNU \file{gdb}.

\lst{game/debug0.mk}{title=\file{/game/Makefile},language=make}

\clearpage
PHONY-цель \file{go}\ идет в \file{Makefile} самой первой, и является главной
целью при запуске \verb|make [go]|. \file{game.elf}\ будет автоматически
перекомпилирован если вы меняли или удаляли какие-то зависимые файлы (с
исходным кодом), затем будет запущен QEMU в режиме \term{отладочного сервера}.
Режимы работы задаются с помощью опций командной строки:

\begin{description}[nosep]
\item[-kernel game.elf] QEMU умеет загружать напрямую ядра операционных систем и
bare metal программы, в которых есть загрузочный заголовок Multiboot
\ref{multiboot}. Для запуска игр скомпилированных в ELF файлы вам не нужны
образы дисков\ --- все ресурсы игры вкомпилированы внутрь исполняемого
кода, и загрузчик ОС сам позаботится чтобы они были помещены в правильные
области ОЗУ.
\item[-nographic] отключить открытие окна эмулятора видеовывода VGA
\item[-s] включить \term{gdb-сервер}
\item[-S] ожидать подключения отладчика до запуска программы
\end{description}

\clearpage
\begin{framed}\noindent
Умение пользоваться \term{отладчиком}\ --- \emph{критический навык} для
любого разработчика встраиваемых систем. 
\end{framed}
\noindent
Вы это поймете как только соскочите с Одурины \ref{arduino}\ на любую другую
среду разработки для микроконтроллеров. Вместо того чтобы долбить команды
отладочного вывода на UART, через (аппаратный) отладчик вы получаете полноценный
интерактивный доступ ко всем внутренностям микроконтроллера или любой программы,
запущенной под операционной системой.
\note{
Если какой-то великий специалист пытается вам усиленно доказывать, что для
разработки отладчик не нужен, и достаточно UART-команд устройства и отладочного
вывода, поздравляю\ --- вы наняли само\-учку-дяйвайщика. Такой ``специалист''
подобен электронщику который настраивает схему с помощью китайского тестера и
лампочки вместо многоканального осциллографа и логического анализатора. Самое
смешное, что таких инженеров массово выпускают профильные технические кафедры
российских ``ВУЗ''ов, в нагрузку 146\% вы также получаете незнание о
существовании систем контроля версий, документирования кода, багтрекинга,
тестирования, и ведения проектной документации.
}

% \clearpage
\url{http://www.linuxcenter.ru/lib/books/linuxdev/linuxdev9.phtml}

\url{https://eax.me/gdb/}

\medskip\noindent
Запуск отладчика состоит из двух частей:
\begin{description}[nosep]
\item[qemu] запускается в фоновом процессе, ожидая подключения на порту TCP
\file{127.0.0.1:1234} как \term{gdb-сервер}
\item[gdb] \emph{отладчик является (сетевым) \term{gdb-клиентом}}
опция \verb|-x game.elf.gdb| указывает скрипт с командами инициализации
отладочной сессии:
\end{description}
\lst{game/gdb0.gdb}{title=\file{/game/game.elf.gdb}}

\clearpage
\secrel{Запуск отладчика с графической оболочкой}

В \linux\ доступна интересная \term{отладочная оболочка} \file{ddd}, которая
обеспечивает не только работу любого gdb-клиента в графическом режиме, но и
\emph{отображение структур данных отлаживаемой программы в виде диаграмм}.

\lst{game/debug1.mk}{title=\file{/game/Makefile},language=make}

\clearpage
\fig{game/debug1A.png}{width=\textwidth}
\clearpage
\fig{game/debug1B.png}{width=\textwidth}
\clearpage
\fig{game/debug1C.png}{width=\textwidth}

\clearpage
\secrel{Отладка прошивки на микроконтроллере}

В случае микроконтроллера схема отладки сильно усложняется:
\bigskip

\fig{/tmp/debug.pdf}{width=\textwidth}
\clearpage

\begin{description}
\item[simulator] программный симулятор, взаимодействует с gdb-клиентом по
протоколу gdb (установки точек останова, пошаговая отладка, выдача информации
по регистрам и т.п.), именно эту схему мы используем с QEMU
\item[OpenOCD] ПО для взаимодействия с различными аппаратными
отладочными адаптерами JTAG/SWD, в том числе включает программную реализацию
JTAG для универсальных конвертеров USB/GPIO (FTDI).
\item[texane] специализированный пакет для взаимодействия с программаторами
STlink
\item[stub] программный компонент, \emph{выполняемый на устройстве как часть
прошивки} микроконтроллера, обеспечивает отладку по аппаратным интерфейсам
общего назначения (UART, Ethernet, TCP/IP,..) без применения специализированных
аппаратных отладчиков.
\end{description}

Вся магия \term{удаленной отладки} реализуется за счет использования
\term{отладочного протокола gdb}: достаточно любым доступным способом
реализовать двусторонний обмен данными между клиентом на рабочей станции
разработчика, и сервером подключенным к отлаживаемой системе.

В результате мы получаем возможность не только загружать прошивку, и отлаживать
ее, но и \emph{взаимодействовать напрямую с аппаратурой}: писать/читать
произвольные области памяти, менять состояние регистров, загружать и выполнять
произвольный код, выполнять автоматизированное тестирование аппаратуры и т.п.

\bigskip
Со стороны программиста мы тоже имеем большую гибкость: можно использовать
графический интерфейс, предоставляемый средами разработки, пользоваться
отдельными отладочными оболочками типа \file{ddd}, самостоятельно
реализовать удобный вам интерфейс, или писать скрипты автоматизации.

\secup
