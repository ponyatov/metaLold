% e-book
% Universal LaTeX headers for e-book publications
\documentclass[oneside,10pt]{book}
%% mobile phone optimized
\usepackage[paperwidth=118.8mm,paperheight=68.2mm,margin=2mm]{geometry}
%% font setup for screen reading
\renewcommand{\familydefault}{\sfdefault}\normalfont
%% hyperlinks pdf style
\usepackage[unicode,colorlinks=true]{hyperref}
%% fix heading styles for tiny paper
\usepackage{titlesec}
\titleformat{\chapter}{\Large\bfseries}{\thechapter.}{1em}{}
\titleformat{\section}{\large\bfseries}{\thesection.}{1em}{}
%% fix first blank page
\usepackage{atbegshi}% http://ctan.org/pkg/atbegshi
\AtBeginDocument{\AtBeginShipoutNext{\AtBeginShipoutDiscard}}
% graphics
\usepackage[pdftex]{graphicx}
\newcommand{\fig}[2]{\noindent\includegraphics[#2]{#1}}

%% bibliography
%\usepackage{titlesec}
\newcommand{\bibfig}[1]{\fig{#1}{height=.56\textheight}}

% xcolor fixes
\usepackage{xcolor}
\definecolor{red}{rgb}{0.7,0,0}
\definecolor{green}{rgb}{0,0.6,0}
\definecolor{blue}{rgb}{0,0,0.7}
\definecolor{magenta}{rgb}{0.7,0,0.7}

% Cyrillization
%% \usepackage[T1,T2A]{fontenc}
\usepackage[utf8]{inputenc}
%% \usepackage[cp1251]{inputenc}
\usepackage[english,russian]{babel}
\usepackage{indentfirst}

% relative sectioning
\usepackage{ifthen}
\newcounter{secdepth}\setcounter{secdepth}{0}
\newcommand{\secup}{\addtocounter{secdepth}{1}}
\newcommand{\secdown}{\addtocounter{secdepth}{-1}}
\newcommand{\secrel}[1]{
\ifthenelse{\equal{\value{secdepth}}{0}}{\part{#1}}{}
\ifthenelse{\equal{\value{secdepth}}{-1}}{\chapter{#1}}{}
\ifthenelse{\equal{\value{secdepth}}{-2}}{\section{#1}}{}
\ifthenelse{\equal{\value{secdepth}}{-3}}{\subsection{#1}}{}
\ifthenelse{\equal{\value{secdepth}}{-4}}{\subsubsection{#1}}{}
}
\newcommand{\secly}[1]{
\section*{#1}
\addcontentsline{toc}{section}{#1}
}
\newcommand{\subsecly}[1]{
\subsection*{#1}
\addcontentsline{toc}{subsection}{#1}
}

% misc
%% [nosep] option in lists/enums
\usepackage{enumitem}

%% typical macros
\newcommand{\email}[1]{$<$\href{mailto:#1}{#1}$>$}
\newcommand{\note}[1]{\footnote{\ #1}}
\renewcommand{\emph}[1]{\textcolor{blue}{\textbf{#1}}}
\newcommand{\cp}[1]{\note{\copyright\ #1}}
\newcommand{\term}[1]{\textcolor{green}{\textit{#1}}}
\newcommand{\termdef}[2]{\textcolor{red}{\textbf{\textit{#1}}}\index{#2}}

% comp
\newcommand{\emc}{$C$}
\newcommand{\cpp}{$C^{+_+}$}
\newcommand{\java}{$Java$}
\newcommand{\py}{$Python$}
\newcommand{\lisp}{$Lisp$}


\author{Dmitry Ponyatov \email{dponyatov@gmail.com} CC BY-NC-ND}
\title{{\Huge \ \\\metal}\\пишем язык (мета)программирования\\
\py\ $\oplus$ \F\ $\oplus$ \prolog}
\date{draft: \today}

\begin{document}

\maketitle
\tableofcontents

\clearpage
\secly{Введение}\secdown

github: \url{https://github.com/ponyatov/metaL}

\bigskip

Бесплатный черновик книги вы можете скачать на странице релизов:

\url{https://github.com/ponyatov/metaL/releases/latest}

\bigskip\noindent
Замечания и комментарии присылайте на e-mail\\или открывайте issue на github.

\subsecly{Отказ от ответственности}

Эта книга посвящена очень \emph{примитивной} реализации того, что особо
грамотные товарищи не захотят назвать языком программирования. Здесь вы не
найдете ленивых лямбд и жутких монад, живущих в лесу Хомского, и прочей
бурбулятристики про обощенный вывод типов. Тем не менее я постараюсь что-нибудь
добавить в отсутствующую нишу русскоязычной литературы по разработке языков
программирования \emph{для самых начинающих}.

\subsecly{Серебряная пуля Брукса}

Сущностью программирования является, прежде всего, не написание инструкций на
конкретном языке программирования, а выработка подробной структуры
взаимодействующих сущностей проблемной области, а также проверка внутренней
непротиворечивости этой структуры.
Следовательно, ни одно средство разработки ПО не сможет существенно снизить
сложность разработки, так как даже если, например, изобрести компьютерный язык,
оперирующий понятиями на уровне проблемной области, программирование все равно
останется сложной задачей, поскольку придется точно определять взаимосвязи между
объектами реального мира, устанавливать исключения, предусматривать все
возможные переходы между состояниями и т.д.

Что делает язык высокого уровня? Они изолируют программу от большей части ее
нецелевой сложности. Абстрактная программа состоит из концептуальных
конструкций: операций, типов данных, последовательностей и коммуникации.
Конкретная машинная программа связана с битами, регистрами, условиями, ветвями,
каналами, дисками и тому подобным. В той степени, в которой язык высокого уровня
воплощает конструкции, которые нужны в абстрактной программе, и избегает всех
нижестоящих, он устраняет целый уровень сложности, который вообще никогда не был
присущ программе. Безусловно, уровень нашего мышления о структурах данных, типах
и операциях неуклонно растет в сторону прикладной области благодаря
возможностям абстракции, предоставляемым ООП, но с постоянно уменьшающейся
скоростью\ --- развитие языков и абстракий движется все ближе и ближе к
прикладной сложности пользователей. Более того, в какой-то момент разработка
языка высокого уровня и фреймворков создает бремя владения инструментом, которое
увеличивает, а не уменьшает интеллектуальную сложность пользователя.

Многие люди ожидают, что достижения в области искусственного интеллекта
обеспечат революционный прорыв, который даст увеличение производительности и
качества программного обеспечения на порядок. При этом не стоит путать
"численный ИИ" как он широко известен сейчас, т.е. нейроные сети и машинное
обучение, с \term{семантическим ИИ}, выполняющим логический вывод \emph{на
основе сетей взаимосвязанных понятий и отношений между объектами}. Наиболее
широко известная технология семантического ИИ\ --- \term{экспертные системы}.

\begin{quotation}\noindent
Экспертная система\ --- это программа, которая содержит обобщенный механизм
логического вывода и базу правил и отношений, принимает входные данные и
предположения, генерурет гипотезы, выводимые из базы правил, дает выводы и
рекомендации, и предлагает объяснение свои результатов для пользователя (путем
отслеживания цепочки логического вывода). Механизмы вывода часто могут иметь
дело с нечеткими или вероятностными данными и правилами, в дополнение к чисто
детерминированной логике.
\end{quotation}

Как эта технология может быть применена к задаче разработки программного
обеспечения?

Работа, необходимая для генерации базы знаний\ --- это работа, которую в любом
случае необходимо будет не просто выполнить единожды, но и постоянно
поддерживать базу знаний в актуальном состоянии. Многие трудности стоят на пути
скорейшей реализации полезных экспертных системных советников для разработчика
программ. Важной частью нашего воображаемого сценария является разработка
простых способов перехода от спецификации структуры программы к автоматической
или полуавтоматической генерации кода, созданию правил тестирования и
диагностики, скриптов развертывания и средств мониторинга. Еще более трудной и
важной проблемой является получение знаний в двух направлениях: поиск четких,
самоаналитических экспертов, которые знают, почему они делают что-то, и
разработка эффективных методов извлечения того, что они знают, и их
использование в базах правил. Необходимым условием для построения экспертной
системы является наличие эксперта.

Самым мощным вкладом экспертных систем, безусловно, должно быть предоставление
на службу неопытному программисту опыта и накопленной мудрости лучших
программистов. Это немалый вклад. Разрыв между лучшей практикой разработки
программного обеспечения и средней практикой очень велик\ --- возможно, больше,
чем в любой другой инженерной дисциплине. Инструмент, который распространяет
передовой опыт, был бы важен.


\subsecly{Языково-ориентированное программирование}

Языково-ориентированное программирование\ --- разработка, опирающаяся на
предметно-специфичный язык (англ. DSL\ --- Domain-Specific Lan\-guage).
Это парадигма программирования, заключающаяся в разбиении процесса разработки
программного обеспечения на стадии
\begin{itemize}[nosep]
  \item 
разработки предметно-ориентированных языков (DSL) и
  \item 
описания собственно решения задачи с их использованием.
\end{itemize}


\clearpage
\subsecly{Метапрограммирование}\label{meta}

\begin{quotation}\noindent
Метапрограммирование — вид программирования, связанный с созданием
\textit{программ, которые порождают другие программы} как результат своей работы
(в частности, на стадии компиляции их исходного кода), либо программ, которые
меняют себя во время выполнения (самомодифицирующийся код).
\end{quotation}

\begin{itemize}
  \item 
\url{https://www.youtube.com/watch?v=QKFrxEkVusg}
  \item 
\url{https://www.youtube.com/watch?v=bt6kU1kuHWA}
\end{itemize}

\noindent
Традиционно при написании программ стараются писать код максимально переносимым
между различными компиляторами. ОС и аппаратурой, для этого создают различные
фреймворки, HAL, стандартные библиотеки и т.п. В итоге вместо быстрых
эффективных программ получаются \textbf{Jаба}троны завернутые в десятки слоев
абстракций и выжирающих ОЗУ гигабайтами\note{Eclipse на запуске на пару минут
вырубает не самый тухлый i7}.
Метапрограммирование через генерацию кода способно решить обратную задачу:
получение исходного кода на \textit{embedded \emc}\ \note{\cpp, \java\ или любом
другом языке, в т.ч. на \py\ для самораскрутки системы}\ максимально
учитывающего все особенности используемой аппаратуры, окружения и конкретной
решаемой задачи. Общая идея\ ---
\begin{itemize}[nosep]
  \item 
\emph{шаблонизация}, 
  \item 
\emph{параметризация} и 
  \item 
\emph{наследование} \textbf{исходного кода}
\end{itemize}
написанного на языках программирования, которые в принипе не знают об ООП,
наследовании и шаблонах (ISO \emc, Makefile, МЭК 61131-3), или не способных их
полноценно реализовать\ \note{интерересно через сколько десятилетий наконец
додумаются встроить в компилятор \cpp\ интерпретатор (\lisp а?) для построения
кода в compile time, вместо сомнительных шаблонов?}. Вся абстрактная каша должна
оставаться на рабочей станции разработчика в высокоуровневом \py-коде,
результат\ --- низкоуровневый код на \emc/LLVM способный работать на сотнях байт
ОЗУ\note{типичное требование для прошивок аппаратуры, сделанных на дешевых
low-end микроконтроллерах, имеюших всего \emph{2+ Кило}байта ОЗУ}.

Идеальным результатом применения metapy будет код, не выполняющий ни одной
машинной инструкции, которая не является необходимой для инициализации
конкретной железки, или решения текущей задачи. Если код должен работать поверх
ОС, в идеале он должен использовать только нативный API и
\term{специфицированный} код
вместо сторонних библиотек и особенно мультиплатформенных фрейморков. В
реальности естественно приходится ограничиваться точечным применением, т.к. есть
legacy код, требования к читаемости выходного кода, обучение программистов
сложной методике, сложность реализации вывода (компиляция мета-моделей), и
невозможность переписать в виде метамоделей весь используемый набор сервисов и
библиотек.

\subsecly{Гомоиконичные языки программирования}\label{homoiconic}

\secup


\part{Обзор и применение \metal}\secdown

Интро еще толком не прописал, поэтому вкратце:
интересует применение экспертных систем для генерации программ для встраиваемых
систем и IoT,

погуглил на тему представления знаний в таких системах, попалась
книжка Марвина Минского (в переводе \cite{minsky}) и пара ссылок с кратким
описанием принципа, подкупает полная поддержка ООП и объектного представления + логический вывод
\begin{itemize}[nosep]
  \item 
прототип решил делать поверх Python\note{чтобы не возиться с управлением
памятью, и использовать несколько удобных библиотек},
  \item 
командный язык а-ля Форт (стек и постфикс) для простоты,
  \item 
как основной инструмент хочется унификацию (как в Прологе) но более
дружественную к императивному программированию,
\clearpage
  \item 
и самое главное гомоиконичность\\
\end{itemize}
(а) чтобы система могла достраивать сама себя (bootstrap) и\\ 
(б) \emph{полностью динамическая интерактивная система а-ля Smalltalk/Self}\\
позволяющая в себя залезть/отладить/модифицировать в рантайме

\bigskip
\noindent
основное прикладное применение: \emph{генерация кода для микроконтроллеров по
шаблонам}\ \note{параметрические куски кода на embedded Си, которые немного
изменяются в зависимости от целевой системы и контекста в котором используются}
наследование дизайна прошивки: есть код прошивки для базового прибора, и
полсотни заказчиков, каждый хочет свои лыжи и гамак, С++ под корпоративным
запретом (и в 2-8К ОЗУ не разбежишься), в итоге исходники неконтролиремо
копипастятся и имеют море наслоений legacy

\secrel{Первые шаги}\secdown

Перед вами книга, посвященная созданию очень примитивных языков программирования
на \py. Здесь вы не найдете ленивых лямбд и жутких монад, живущих в лесу
Хомского, и прочей бурбулятристики про обощенный вывод типов.
И все же мне хочется пошагово показать создание собственной реализации языка
программирования, более близкого по ощущениям к Self, Smalltalk и \lisp.
В реализациях этих языков вы можете \emph{программно} строить части программ во
время исполнения, вмешиваться в процесс работы ядра языка, и добавлять в язык
различные возможности других языков, например смешивать фнукциональное,
императивное и логическое программирование.

\begin{quotation}\noindent
Метапрограммирование — вид программирования, связанный с созданием
\textit{программ, которые порождают другие программы} как результат своей работы
(в частности, на стадии компиляции их исходного кода), либо программ, которые
меняют себя во время выполнения (самомодифицирующийся код).
\end{quotation}

\begin{itemize}
  \item 
\url{https://www.youtube.com/watch?v=QKFrxEkVusg}
  \item 
\url{https://www.youtube.com/watch?v=bt6kU1kuHWA}
\end{itemize}

Такие богатейшие возможности \term{метапрограммирования} возможны благодара
тому, что эти языки \term{гомоиконичны}: их реализацации работают как
\emph{живая интерактивная система} используючая структуры данных как
представление программы и исполняемый код.

\begin{quotation}\noindent
\term{Гомоиконичность} (гомоиконносль, англ. homoiconicity, homoiconic)\\
свойство некоторых языков программирования, в которых \emph{представление
программ является одновременно структурами данных} определенных в типах самого
языка, \emph{доступных для просмотра и модификации}. Говоря иначе,
гомоиконичность\ --- это когда исходный \textit{код программы} пишется
\textit{как базовая структура данных}, и язык программирования знает, как
получить к ней доступ (в том числе в рантайме при работе программ у конечного
пользователя).
\end{quotation}

Реализация \term{виртуальной машины} гомоиконичного языка 

\clearpage
\paragraph{Применение}\ \\ \bigskip

\begin{itemize}[nosep]
\item \emph{обработка текстовых форматов данных}\\
	файлы САПР, исходные данные для расчетных программ
\item командный интерфейс для устройств на микроконтроллерах\\
	управление человеко-читаемыми командами, \emph{передача пакетов данных
	любой структуры и типов}
\item реализация специализированных скриптовых языков
\item обработка исходных текстов программ\\
	модификация, трансляция на другие языки программирования,\\ 
	\emph{универсальный язык независимых от языка шаблонов и метапрограммирования}
	для ЯП с ограниченными или отсутствующими макросами
\end{itemize}

\secrel{Установка}\label{install}

\url{https://github.com/ponyatov/metaL/releases/latest}

\begin{verbatim}
~$ git clone [-b master] https://github.com/ponyatov/metaL.git
~$ cd metaL
~/metaL$ python ./metaL.py
\end{verbatim}

Интерпретатор написан на диалекте \py 2, также в системе должны быть установлены
библиотеки

\begin{verbatim}
~$ sudo pip install --upgrade pip
~$ sudo pip install ply
\end{verbatim}

Для использования веб-интерфейса

\begin{verbatim}
~$ sudo pip install flask wtforms
\end{verbatim}

\clearpage
Если вы не хотите тащить большую систему, и вас интересует только базовый язык
\metal, можно установить проект Ouroboros \ref{ouro}:\ \note{из-за его
минималистичности установку можно свести к ручной загрузке всего нескольких
файлов с GitHub}

\begin{verbatim}
$ git clone -o gh https://github.com/ponyatov/Ouroboros
\end{verbatim}

Для использования под \win\ удобен редактор \vim:\\
\url{https://www.vim.org/download.php}


\secup


\secup

\part{Реализация в деталях}\label{implement}\secdown

\secrel{Концепция фреймов Марвина Мински}\label{frame}\secdown

\clearpage
\cite{minsky} Марвин Минский \textbf{Фреймы для представления знаний}

\begin{itemize}
%   \item 
% \url{https://royallib.com/read/minskiy_marvin/freymi_dlya_predstavleniya_znaniy.html#0}
  \item 
\url{https://ponyatov.quora.com/Minsky-Frames-Database-metaL}\\(см. видео в
начале)
\end{itemize}

В качестве модели представления (мета)программ было выбрано расширенное
представление фреймов Мински. Оригинальные фреймы не имели очень важного для
метапрограммирования функционала: \textit{способности хранить упорядоченные
объекты}. Эта фича необходима для представления любых
программ\note{последовательного набора инструкций, или рекурсивно вложенных
структур}, в качестве примера см. деревья разбора/AST и реализацию атрибутных
грамматик \cite{dragon2}. С другой стороны, фреймы имеют практически полное
соответстивие объектной парадигме, в т.ч. объектам \py.

Если мы попытаемся описать дерево программы через граф объектов (фреймов), мы
сталкиваемся с необходимостью иметь \emph{упорядоченные контейнеры}, например
для хранения операндов в выражении деления. Одновременно нам необходим
\emph{ассоциативный массив} для хранения и обработки \term{атрибутов}\ при
преобразованиях кода с использованием \term{атрибутных грамматик}.

Оригинальная модель фреймов не предусматривает упорядоченное хранение, поэтому
был выбран расширенный вариант модели, для некоторой универсализации,
\begin{itemize}
  \item 
выделенная иерархия классов применяется для отделения логики фреймов от логики
работы объектной системы в Python\ \note{хотя в принципе динамическая природа
\py\ позволяет реализовать все на встроенных механизмах его объектного движка},
  \item 
явные манипуляции с фреймовыми структурами демонстируют принципы реализации на
низкоуровневых языках с жесткой типизацией, AOT-компиляцией и соответственно
невозможностью произвольно менять структуру класса или единичного объекта в
рантайме (\cpp, \java)
  \item 
добавление некоторых фич, характерных для функциональных и логических языков 
программирования \note{унификация/backtracking и структурный pattern matching}
дает возможности, крайне полезные для метапрограммирования и реализации
интеллектуальных систем (базы знаний, экспертные системы, \term{семантический
ИИ}).
\end{itemize}

\lst{lst/frame.py}{language=Python}
 
% , предложенного Марвином Мински,
% добавлением функционала упорядоченного контейнера `nest[]`, позволяющего
% не только хранить `attr{}`ибуты (слоты),
% но и любые элементы данных в явно заданном порядке.
% 

% 
% Также в большинстве случаев у нас есть необходимость хранить для любого
% элемента данных два поля:
% * `type` <br>
% явно указывающий на тип фрейма. Мы принципиально не можем оперировать
% двумя фреймами в выражении типа `<string:> + <number:>` без их приведения к
% одному типу, причем это приведение часто зависит от контекста, в каком именно
% смысле мы это выражение используем (привет долбанутый JavaScript)
% * `value` <br>
% атомарное значение, хранимое в типе языка реализации (Python): нам нужно
% именовать объекты, хранить значение строк и числовых данных, поэтому также
% необходимо подкласс фреймов для представления таких значений-примитивов.
%   
% (*) имена type/value фиксированы требованиями библиотеки PLY, если вы захотите
% использовать ее для создания собтвенного языка метапрограммирования или CLI
% вместо Python

\secup

\clearpage
\secrel{Язык \metal: исправленный \F}\secdown

Хотя мы стараемся уйти от использования языка программирования как основного
средства разработки \ref{nolang}, в любом случае нам нужен способ ввода данных и
систему, и управления вычислениями.

\emph{\metal\ не является языком
программирования}, это \term{командный язык} с помощью которого выполняется
\begin{itemize}[nosep]
  \item 
создание фреймов, 
  \item 
модификация \term{фреймовой базы знаний}, 
  \item 
запуск/останов скриптов и демонов. 
\end{itemize}
Однако очень простое \term{императивное
программирование} может выполняться и на \metal, так как этот язык позволяет
определять новые \F-\term{слова}, и поддерживает \term{конкатенативное
программирование} через разделяемый стек.

\clearpage
В качестве прототипа для \metal\ был выбран язык \emph{\F: это самый
элементарный язык программирования}, который вы только можете найти. Вы можете
самостоятельно написать свой \F\ за пару вечеров или пару недель на любом языке
программирования, и для любого типа компьютера.

% \smallskip\noindent
\F\ был создан в 70х годах Чаком Муром для управления оборудованием
(радиотелескопом), и \emph{\F\ до сих пор великолепен в роли командной оболочки}
(CLI) для подобных задач. В том числе \F\ очень хорошо подходит как командная
консоль для микроконтроллеров с очень небольшими объемами ОЗУ порядка 8-20
Кило(!)байт.

Но в роли основного языка программирования \F\ очень плох:
\begin{description}[nosep]
\item[низкоуровневая модель ВМ языка]: \F\ по факту является ассемблером
\term{виртуальной стековой машины}, и как с любым ассемблером вам приходится
самостоятельно выписывать все фишки, которые в mainstream языках доступны из
коробки в базовой спецификации языка. Как пример, в стандартное ядро языка не
входит поддержка вычислений с плавающей точкой, и полностью отсутствуют средства
динамического выделения памяти.
\item[прямой доступ к памяти по адресам]\ делает код на \F е крайне
нестабильным: ошибки в адресации тут же приводят к перезаписи данных и кода по
случайным адресам. В результате при программировании на \F\ нужно работать с
памятью на порядок аккуратнее, чем на \emc.
\end{description}

\secup

\secrel{Раскрутка языка (bootstrap)}\label{circ}\secdown

В этой книге нам нужно показать всю мощь языка специально заточенного под
метапрограммирования. Лучшим способом для этого является его \term{bootstrap},
или \term{раскрутка}: написать \term{метациркулярную} реализацию языка
программирования\ --- \emph{на нем самом}.

\secrel{Метациркулярный интерпретатор}

\begin{quotation}
Метациркулярный интерпретатор является интерпретатором, написанным в (возможно,
более базовой) реализации того же языка. Обычно это делается для того, чтобы
экспериментировать с добавлением новых функций на язык или созданием другого
диалекта.
\end{quotation}

В целях демонстрации того, как работает язык программирования, в литературе
часто применяют этот метод: некоторые части интерпретатора описываются на том же
языке программирования. Это позволяет не только показать внутреннее устройство,
но и служит реальным примером применения.

Если в комплект поставки включить полную метациркулярную реализацию языка,
пользователь также может адаптировать язык под свои нужды, или написать свой
клон, но для этого должно выполняться одно очень важное, критическое условие\
--- \emph{документация должна поставляться} не как руководство
пользователя, а \emph{как учебник по написанию собственной версии языка}.

\bigskip
Понять метациркулярность компилятора очень просто: у нас есть исходный код
компилятора для некоторого языка программирования, и исполняемый файл этого
компилятора, оба версии N. Исходный код модифицируется, подается на вход
\file{компилятора-N}, в результате получем исполняемый код
\file{компилятора-N+1}. Для тестирования новой версии мы еще раз подаем
исходный код N+1 на вход \file{компилятора-N+1}, и он собирает сам себя. Такой
способ в частности применяется при сборке GCC из GNU Compilers Collection.

\bigskip
Для интерпретаторов динамических языков используется другой способ \cite{plai},
похожий на то как мы написали всю внутреннюю механику \hico\ на \py: 

\secup
\secrel{Web-интерфейс /Flask/}\label{web}\secdown
\secup

\secrel{Элементы языка \prolog}\secdown

\begin{itemize}[nosep]
  \item 
\url{http://yieldprolog.sourceforge.net/tutorial1.html}
  \item 
\url{http://yieldprolog.sourceforge.net/tutorial2.html}
\end{itemize}

\secrel{Магия алгоритма унификации}\secdown

\secrel{Генераторные функции и yield}\label{yield}

Ключевое слово \file{yield}\ в \py\ превращает любую функцию, в которой оно
используется, в функцию-\term{генератор}. Вызов генератора вместо
выполнения функции возвращает объект-\term{итератор}. Если его использовать в
качестве параметра цикла \file{for}, или явно вызывать встроенй метод
\verb|__next__()|, то вы сможете использовать \term{ленивые вычисления}\ в
обычной императивной программе на \py.

\begin{quotation}\noindent
\term{Ленивые вычисления} (англ. lazy evaluation, также отложенные вычисления)\
--- применяемая в некоторых (функциональных) языках программирования стратегия
вычисления, согласно которой вычисления следует откладывать до тех пор, пока не
понадобится их результат.
\end{quotation}

В рамках \py\ полная реализация ленивых вычислений недоступна \ref{lazy}, тем не
менее использование генераторов позволяет вычислять функции в бесконечном цикле,
возвращая промежуточные результаты. Также на генераторных функциях построен
механизм \term{логического вывода в возвратами}, используемый в языке \prolog,
который мы рассмотрим далее.

\medskip
\lst{prolog/00.py}{language=Python}

Генератор \file{person()}\ соответствует 0-арному \term{отношению}
\file{person()}, которое определяет свойство быть человеком (person) для
некоторых внешних объектов, которые явно не указаны в качестве параметров
отношения.

\begin{quotation}\noindent
\term{отношение}\ --- свойство некоторого объекта, или связность нескольких
объектов между собой.
\end{quotation}

\begin{quotation}\noindent
\term{арность}\ --- число объектов: параметров отношения
\end{quotation}

\begin{quotation}\noindent
\term{предик\'{а}т} (n-местный, или \term{n-арный})\ --- это логическая функция
с множеством значений \{0,1\} или \{false,true\} (\{ложь, истина\}),
определённая на множестве $M=M_1 \times M_2 \times \ldots \times M_n$. Таким
образом, каждый набор элементов множества $M$ характеризуется либо как
``истинный'', либо как ``ложный''.
\end{quotation}
Предикат можно связать с математическим \term{отношением}: если кортеж\\
$(m_1,m_2,\dots ,m_n)$ принадлежит отношению, то предикат будет возвращать на
нем \file{true}. В частности, одноместный предикат определяет отношение
принадлежности некоторому множеству.

\begin{description}[nosep]
\item[унарное отношение] \file{relation(object)}\\
определяет некоторое свойство объекта, задает \term{множество} объектов,
обладающих этим свойством, и соответствует одноименной функцию-предикату
способную проверить обладает ли \file{object} заданным свойством \file{relation}
\item[бинарное отношение] \file{binar(obj1,obj2)}\\
связывает два объекта
\item[n-арное отношение] \file{Nary(obj1,\ldots)}\\
\verb|sum(A,B,product)| задает \term{тернарное} отношение суммы: \verb|product|
является суммой \verb|A| и \verb|B| (порядок параметров предиката важен, но не
предопределен)
\item[нуль-арное отношение] \file{time()}\\
обобщение, отношение заданное для внешних, явно не указанных, неименованных
объектов; например текущее время, или состояние системы. Такие объкты можно
задавать и логической переменной, но описание отношения принадлежности строк
файлу \verb|FileName()| получится значительно сложнее, многословнее, и скорее
всего с использованием рекурсии.
\end{description}

\bigskip
Генератор (отношение) может быть задан и для бесконечного множества значений,
например бесконечной последовательности, и как раз здесь срабатывает принцип
ленивых вычислений: каждое новое значение вычисляется по необходимости, в нужный
момент, при этом не требуется\note{потенциально бесконечная, или
непредсказуемая} резервация памяти для хранения данных.

\medskip
\lst{prolog/01.py}{language=Python}
\lst{prolog/02.py}{language=Python}

\secrel{Логические переменные}

Переменные в языке \prolog\ используются в процессе \term{логического вывода}
как средство передачи значений между многими предикатами. Попробуем аналогично
использовать переменную \py:

\medskip
\lst{prolog/var03.py}{language=Python}

То же самое в системе фреймов: заводим новый класс переменной:

\lst{prolog/var04.py}{language=Python}

Генераторная функция \term{связывает переменную} со значением, и
\emph{возвращает переменную} как результат на каждой итерации:

\lst{prolog/var04.out}{}

\secup
\secup
\secrel{Динамическая компиляция}\label{dyna}\secdown

\secrel{LLVM}\label{llvm}\secdown
\url{https://www.youtube.com/watch?v=q6uF3a-SJUU}
\secup


\secup

\secrel{загрузчик кода embedded \cpp)

\url{http://port70.net/~nsz/c/}
%\url{http://eli-project.sourceforge.net/c_html/c.html}

\lst{../codein/codein.rc}{title=codein.rc}


\secup

\secrel{Применение для встраиваемых систем}\label{mcu}\secdown
\secrel{Микроконтроллеры}\secdown

\secrel{Общая архитектура RISC микроконтроллеров}\secdown
\secrel{Процессорное ядро}\label{cpucore}

RISC

\secrel{Управление памятью}

Модель памяти WebAssembly очень проста. Это плоский «кусок» памяти, в котором
находится код программы, глобальные переменные, стек и куча. Есть возможность
сделать так, чтобы память была расширяемой, то если если при очередном выделении
памяти нам не хватает места, то верхняя граница памяти автоматически
увеличивается.
Весь блок памяти доступен из JavaScript как на чтение так и на
запись как массив байт.

\fig{web/wasm/memory.png}{width=\textwidth}


\secrel{WDT: сторожевой таймер}

Cторожевой таймер предназначен для защиты от аппаратных сбоев, приводиящих к
зависанию МК. Реализуется в виде таймера, срабатывание которого приводит к
аппаратному сбросу.

\secup

\secrel{Одурино}\label{arduino}

Популярная платформа для начинающих, с самым поганым соотношением
цена/возможности

\secrel{MSP430}\label{msp}

16-битные МК средней мощности для устройств с батарейным питанием, неплохо
подходят как промежуточный этап перехода к полноценным микроконтроллерам.

\secrel{ARM Cortex-M /STM32/}\label{cortex}\label{stm}\secdown

\secrel{Настройка рабочего стенда (железо и ПО)}

\url{https://www.youtube.com/watch?v=DCjhJCk_bIE}

\href{https://www.aliexpress.com/item/FREE-SHIPPING-ST-Link-V2-stlink-mini-STM8STM32-STLINK-simulator-download-programming-With-Cover/1814606455.html}{\fig{mcu/arm/cnstlink.jpg}{height=.65\textheight}}
\href{https://www.aliexpress.com/item/48-MHz-STM32F030F4P6-Small-Systems-Development-Board-CORTEX-M0-Core-32bit-Mini-System-Development-Panels/32831635311.html}{\fig{mcu/arm/stm32f040.png}{height=.65\textheight}}

Минимальный комплект\ --- китайский клон программатора ST-Link v2 и отладочная
плата на базе самого мелкого и дешевого микроконтроллера STM32F030F4P6 (55р в
розницу в самом жлобском магазине).


\secup
\secup

\secrel{embedded Linux}\label{linux}

\secrel{Событийная архитектура вместо ОСРВ}\label{event}\secdown

\clearpage
Разработчики встроенного программного обеспечения из разных отраслей независимо
друг от друга заново открывают шаблоны для создания параллельного программного
обеспечения, которое является более безопасным, более гибким и более простым для
понимания, чем обычные потоки и различные механизмы блокировки
ОСРВ\note{Операционная Система Реального Времени}.

Эти лучшие практики всегда предпочитают \emph{управляемые
событиями}\note{\term{event-driven} событийное программирование} асинхронные
неблокирующие инкапсулированные \term{активные объекты}, каждый из которых
управляется внутренним \term{конечным автоматом}, а не чистыми блокирующими
потоками RTOS.

В следующих разделах объясняются концепции, связанные с этим набирающим
популярность \term{реактивным} подходом, и, в частности, как он применяется к
встроенным системам реального времени, на которые ориентированы платформы и
инструменты Quantum Leaps \cite{psicc2}.

Практически все встроенные системы по своей природе \term{реактивны}, что
означает, что их основная задача\ --- реагировать на события, такие как нажатия
кнопок, касания экрана, тайм-ауты или прибытия некоторых пакетов данных.
Следовательно, большую часть времени встроенная система ожидает событий, и
только после распознавания события система реагирует, выполняя соответствующие
вычисления.

Две основные проблемы такого подхода:
\begin{itemize}[nosep]
\item выполнить правильные вычисления 
\item и выполнить их \emph{своевременно}.
\end{itemize}

\secrel{QP\texttrademark\ Real-Time Embedded Frameworks}\label{qp}

{\Huge \href{https://www.state-machine.com/}{$QuantumL^{e_a}Ps$}\ \cite{psicc2}}

\lst{event/game.c}{language=C}

\secup

\secrel{CODEin: таблетка от legacy}\label{codein}\secdown

\noindent
Огромной проблемой поддержки уже существующих проектов является практически
полное отсутствие бесплатных средств интерактивного анализа исходного кода.
Каким бы суперкрутым метапрограммистом вы не были, вы обязательно вляпаетесь в
legacy код, прибитый ржавыми гвоздями к огромной табличке: ``Ничего не трогать!
Только поправить''.

\secrel{IDE: контуженные Борландом}\label{ide}\secdown

\fig{code/VS.png}{height=.31\textheight}
\fig{code/turboC.jpg}{height=.31\textheight}
\fig{code/IAR.png}{height=.31\textheight}

\clearpage
Если считать начиная с первых версий языка \st\ и рабочих станций компании
Xerox, история интегрированных сред разработки (IDE) насчитывает уже больше 40
(!) лет. И за это время прогресс IDE так и застрял на уровне мультиоконников
в стиле Borland.

Впрочем, одна вменяемая среда все же есть\ --- Emacs, но к
сожалению \term{средства расширения пользователем} используют довольно
специфичный \lisp.

Другая \emph{IDE-платформа} \eclipse\ по задумке выглядела перспективной, но
расширение с помощью \java\ убивает всю идею сложностью вхождения пользователю,
которому нужно только чуть-чуть поправить поведение типового редактора, или
добавить простую визуализацию \ref{ideviz}.

Хорошей базой для разработки средств разработки\note{метаIDE, или
\term{метасреда}} могут быть современные \st-системы типа
\href{https://pharo.org}{Pharo}, сочетающие легкий для освоения скриптовый язык,
встроенный интерактивный отладчик, и богатый GUI. К сожалению ценовая политика
поставщиков реализаций \st\ убила великолепный язык программирования, поэтому
его развитие почти остановилось из-за очень маленького пользовательского
сообщества.

\secrel{Проблемы традиционных IDE}\secdown

\secrel{Фиксация на файловом представлении}

Разработка фиксируется на редактировании файлов исходного кода, полностью
игнорируя критическую проблему\ --- \emph{необходимость передачи знаний между
разработчиками} \ref{litprog}.

Использование файлов хорошо совместимо с mainstream хранением публичных проектов
в репозиториях на GitHub, и необходимо для работы любых компиляторов, но по
факту \emph{необходимо хранение в форматах представления знаний}\ --- объектные
СУБД, семантические сети и т.п.

\secrel{Необходим фронтенд компилятора}

Для работы с исходным кодом необходима реализация большой части фронтенда
компилятора: синтаксический разбор для подсветки синтаксиса, построение таблиц
символов для ссылок и переходов по исходному тексту, препроцессор для скрытия
неиспользуемого кода, completion по программным обхектам, подстветка
синтаксических ошибок,\ldots

\secrel{MDI стиль интерфейса}

\secrel{Полностью отсутствуют средства визуализации}\label{ideviz}

\secrel{Расширение пользователем усложнено}

\secup

\clearpage
\secrel{Критически необходимо Literate Programming}\label{litprog}

\href{https://en.wikipedia.org/wiki/Literate_programming}{Литературное
программирование}\ (LP) предложено Доналдом Кнутом как решение проблемы
\emph{передачи знаний между разработчиками}. Традиционная разработка уже 60 лет
фиксируется на редактировании файлов исходного кода. В ранние года развития ЭВМ
такой подход вполне понятен\ --- вычислительных ресурсов едва хватало. Но этот
явный косяк в подходах не был увиден и решен даже в конце 90х, когда объемы
ОЗУ исчислялились мегабайтами, и жесткий диск достаточных объемов стоял в каждом
персональном компьютере.

Сейчас широко используются различные костыльные решения типа \file{Doxygen},
\file{javadoc} и т.п., но их функционал явно недостаточен для полноценного
документирования. Они вполне применимы для справочников по API библиотек, но
документирование требует применение гипертекста, диаграмм, а иногда и сложной
математической верстки.

Парадигма LP предполагает полное объяснение логики программы на естественном
языке, в формате пояснительной записки, в которую включаются фрагменты исходного
кода, передаваемого компилятору при сборке программы. \emph{При документировании
ПО необходим функционал сквозной увязки объектов в исходном коде, и элементов
документации}\ --- такая увязка при полноценном применении LP должна
обеспечиватся представлением проекта как \term{документной базы знаний}\note{с
поддержкой перекрестных ссылок между документацией и комментариями в исходном
коде, построением индексов объектов и терминов, нечетким поиском, визуализацией
структуры программ, и т.п.}.

К сожалению, классическое LP оказалось неприменимо из-за важной особенности\ ---
\emph{каждый фрагмент исходного кода (\term{блок программы}) должен быть описан
полностью}, так как на компиляцию он передается целиком. На практике такой
подход неприменим: мы должны полностью увязать текст документации с каждым
программным блоком не только по количеству использований (только один раз), но
\emph{и по порядку}\ --- декларация функций и переменных должна идти до их
первого использования.

В то же время, \emph{документация на программный продукт\note{или
программно-аппаратный комплекс\ --- в этой книге это подразумевается, так как мы
говорим о встраиваемых системах}\ принципиально нелинейна}: вы это легко сможете
увидеть сами, если попытаетесь написать книгу по какой-нибудь достаточно сложной
программе. Книга предполагает линейное чтение, компоненты программы описываются
с разных точек зрения и в разных местах текста.
\begin{itemize}[nosep]
\item
В руководстве пользователя рассматривается интерфейс, и стыковка с внешними
системами.
\item
В руководстве программиста\note{для пользователя который планирует расширять
систему самостоятельно}\ --- API и внутреннее устройство, причем основная
реализация функция может быть описана в одном разделе, а код обеспечивающий
безопасность той же функции в другом.
\end{itemize}

Кнут предполагал уйти от написания программ с точки зрения компьютера, и
позволить программистам разрабатывать программы способами и очередностью кода,
определяемыми ходом мышления разработчика.

Пакет \file{WEB}, который использовал Кнут, работал в пакетном режиме, читал
``литературный'' код, и генерировал из него как файлы исходного кода для
компилятора, так и файлы на языке разметки документации\note{\TeX}.
В практическом смысле реализация LP должна быть сделана в IDE в виде
интерактивного просмотра документации. Изменение блоков кода должно отображаться
немедленно, или при обновлении страницы документации по команде.

Кнут утверждает, что литературное программирование обеспечивает первоклассную
систему документирования, которая не является надстройкой над процессом
разработки, а наоборот естественным образом направляет процесс разработки через
изложение своих мыслей при создании программы. Кнут описывает разработку как
\emph{построение сети абстрактных концептов} о содержании и функционале
программы, что напрямую пересекается с концептуальным программированием
\cite{tyugu} как построение абстрактной модели предметной области.

Первичность документации перед исходным кодом, и описание компонентов программы
в визуально-представимом виде стимулируют разработчика не только писать краткие
записи по вносимому коду, но и синхронизировать описание с модифицируемым кодом.
Традиционно для этого используются комментарии в коде, но система интерактивного
ввода позволяет набрасывать краткие записи\note{для таких sticky-записей вполне
подходят не только векторные диаграммы с активными элементами-ссылками на
программные элементы, но и просто ``кроки'' нарисованные мышью от руки}\ по ходу
работы, что в дальнейшем спасет legacy-разгребальщика от многодневных сессий в
отладчике и тонно-литров кофе в качестве антидепрессанта.


\secup

\secrel{Универсальная IDE на wxPython}\label{wx}\secdown

\noindent
Использование приложений с нативным GUI стало немодно: веб-интер\-фейс
обеспечивает б\'{о}льшую гибкость, и возможность
интеграции с внешними сетевыми сервисами (GitHub, Google). Если рассматривать
требования браузера по памяти по отрисовке GUI, и быстродействия интерфейса,
вопрос о применении \term{нативного GUI} остается открытым.

\secrel{Установка под \win}

\begin{verbatim}
$ python-2.7.16.msi
$ pip install --upgrade pip
$ pip install --upgrade ply
$ wxPython3.0-win32-3.0.2.0-py27.exe
\end{verbatim}

\noindent
Поддержка версии 2.7 закрыта, поэтому часть пакетов придется ставить не через
\file{pip}, а индивидуальными инсталляторами.

\secrel{Базовый мультиоконный GUI}

\clearpage
\lst{gui/wx00.py}{language=Python,title=gui/wxide.py}

\begin{description}
\item[ide]
\item[ideWindow]
\item[ideConsole]
\end{description}

\secup
\secrel{emCin: загрузка кода на embedded \emc}\label{emcin}

Даже если каким-то чудом вам удастся найти триалку коммерческого интерактивного
анализатора кода или \href{https://www.sourcetrail.com/}{Sourcetrail}, внезапно
выяснится что:
\begin{itemize}[nosep]
  \item диалект \emc\ исходников вашей прошивки не поддерживается,
  \item при попытке загрузки проекта вываливается 100500 сообщений об ошибках
  синтаксиса
  \item анализатор умеет работать только с \cpp,
  \item ничего не знает про особенности кодирования под микроконтроллеры, и
  \item неспособен загрузить исходный код ядра Linux с учетом всей пары сотен
  настроек конфигурации, и уж тем более
  \item не имеет никакого понятия о Makefile, autohell, файлах проектов IAR и
  двух десятках сборочных .batников, наклёпанных кем-то из пяти ваших
  предшественников.
\end{itemize}

\clearpage
Короче, вы попали.
\bigskip

Можете не надеяться что, прочитав эту главу, вы сможете наклепать
супер-пупер-анализатор с рефакторингом и автогенератором тестов. Чтобы написать
инстумент, способный хотя бы обеспечить приличную навигацию по исходному коду,
нужна команда разработчиков, по квалификации стремящаяся к JetBrains.

Максимум что я могу вам предложить\ --- сделайте пару пробных шагов, вдруг вы
загонитесь, и запустите проект по разработке CodeShit Studio.

\secrel{загрузчик кода embedded \cpp)

\url{http://port70.net/~nsz/c/}
%\url{http://eli-project.sourceforge.net/c_html/c.html}

\lst{../codein/codein.rc}{title=codein.rc}



\secup

\secrel{mets: генератор метаОС\\для встраиваемых систем}\label{os}

Существует огромное количество операционных систем\note{встраиваемых, реального
времени, специализированных, общего назначения, распределенных, защищенных,
исследовательских, монолитных и микроядерных,\ldots}, и систем похожих на них по
поведению \ref{qp}, в большей или меньшей степени разделяющих одни и те же
компоненты, библиотеки, и принципы дизайна.
В этом разделе предлагается альтернативных подход на базе принципов
метапрограммирования: автоматизированное построение средств runtime-поддержки
для конкретного проекта, и адаптивное конфигурирование в зависимости от
набора фич, которые вам необходимы.

\secup

\secrel{\A\ органайзер}\label{android}\secdown

\noindent
\begin{tabular}{l|p{8.3cm}}
\tfig{android/android_plan.png}{height=.45\textheight} &
Мобильниый телефон\ --- компьютер, который всегда с собой. Но \emph{удобных
средств программирования}, работающих на мобильном \textit{телефоне},
практически нет.
Разработка под \A\ и реализация on-device системы программирования\ --- тема
отдельной большой книги. Для начала можно попробовать написать \emph{органайзер,
программируемый пользоваталем}
\\ \end{tabular}

\medskip\noindent
На планшете есть несколько сред разработки для \emc, \py\ и \js, но на
телефоне они совершенно неюзабельны.

\clearpage
\begin{description}
\item[PIM] Personal Information Manager

Функции, выполняемые органайзером\note{персональным информационным
менеджером}:
\begin{itemize}[nosep]
  \item 
\emph{планирование задач} для контроля за их самостоятельным и сторонним
выполнением (ToDo list, task-трекер, мобильный CRM);
  \item 
планирование событий, привязанных к датам и времени (праздники или встречи);
  \item 
\emph{напоминальники и зудильники} об определённых пользователем событиях;
  \item 
управление контактами (адресно-телефонная книга);
  \item 
записная книжка и листки-липучки;
  \item 
личные записи (дневник);
  \item 
интеграция с электронной почтой и мессенджерами
  \item 
\emph{персональная база знаний}.
\end{itemize}

\item[PPS] Personal Planning System, система персонального планирования\\
специально заточенная на трекинг задач, с точки зрения конкретного
человека.
\end{description}

\noindent
Несмотря на десятки лет усилий, даже такие гиганты как Google и Microsoft не
смогли решить проблему создания полноценного органайзера, в который by design
должен был превратиться смартфон:
\begin{itemize}
  \item доступны только примитивные типы задач, при этом на практике нужно
  множество вариантов, от простого будильника, встречи, действий привязанных по
  месту, периодические задачи с разным масштабом\note{от 30 минут для отдыха
  глаз, до года для дней рожденья}, до задач характерных для систем
  groupware
  \item полностью отсутствует функционал трекинга проектов: групповые
  задачи, зависимости задач по времени, исполнителям и ресурсам, делегирование и
  контроль,\ldots
  \item отсутствие средств индивидуальной адаптации, включая средства
  программирования пользователем, и доступ к внешним приложениям, библиотекам и
  сенсорам
\end{itemize}

\noindent
Для таких внутренне сложных приложений как органайзер, планировщик или трекер
задач, подход традиционных приложений с пользовательским GUI не подходит. Чем
больше автор приложения усложняет его, добавляя все новые и новые функции, тем
сложнее для пользователя становится его освоение. При этом даже для
очень переусложненного органайзера обязательно встретится случай использования,
под который не подходит ни один из предусмотренных в приложении вариантов.
Например потребуется задача, которая должна срабатывать одновременно по времени,
местоположению, и условию выполнения другой задачи, назначеная другому
пользователю в вашей рабочей группе.

Если наборот в дизайне органайзера пойти от архитектуры, построенной на
минимальном ядре, расширяемом пользователем через написание скриптов, мы снижаем
начальный порог вхождения (за счет уменьшения необходимых усилий для освоения),
одновременно предоставляя пользователю механизм персональной адаптации.

\medskip\noindent
\begin{tabular}{l p{8.7cm}}
\tfig{android/basact.png}{height=.6\textheight} &
Для создания проекта в Android Studio лучше всего подходит Basic Activity, так
как из коробки в графическом интерфейсе приложения есть элемент добавления новой
задачи.

\medskip
\begin{tabular}{l l}
Name & HICO \\
Package name & \file{io.github.ponyatov.hico} \\
Save location & \file{/home/ponyatov/hico/Android} \\
Language & Java \\
Minimum API Level & API14\\
\end{tabular}
\\
\end{tabular}

\bigskip\noindent
\begin{tabular}{l p{9.5cm}}
&\\
\tfig{android/planning.png}{height=.2\textheight} &
Для начала стоит заменить иконку приложения, также создав комплект round icon
для совместимости со новыми версиями \A\ использующими ``пузырчатый'' интерфейс.
\\
\end{tabular}

\bigskip\noindent
Для подготовки иконок нужно иметь некоторые навыки для работы с графикой и
каким-то из графических редакторов, типа \href{https://www.gimp.org}{GIMP}. Если
у вас уже есть черновик иконки, можете воспользоваться визардом по адресу\\
\url{http://jgilfelt.github.io/AndroidAssetStudio/icons-launcher.html}\\
Загрузив в него черновик иконки, и включив эффект Shape:Bewel, вы сможете
скачать \file{ic\_launcher.zip} содержащий готовый набор иконок для разных
разрешений экрана. Копировать их в проект придется по одному файлу вручную, так
как в новых версиях Android Studio изменилась схема именования каталогов с
ресурсами: с \file{drawable\_}\ на \file{mipmap\_}.

\bigskip
\begin{lstlisting}
~/hico$ find Android/ -type f \
	-regex .+ic_launcher.png$ -exec file {} +
	
Android/app/src/main/res/mipmap-xhdpi/ic_launcher.png:   
	PNG image data, 96 x 96, 8-bit/color RGBA, non-interl
Android/app/src/main/res/mipmap-xxhdpi/ic_launcher.png:  
	PNG image data, 144 x 144, 8-bit/color RGBA, non-inte
Android/app/src/main/res/mipmap-mdpi/ic_launcher.png:    
	PNG image data, 48 x 48, 8-bit/color RGBA, non-interl
Android/app/src/main/res/mipmap-xxxhdpi/ic_launcher.png: 
	PNG image data, 192 x 192, 8-bit/color RGBA, non-inte
Android/app/src/main/res/mipmap-hdpi/ic_launcher.png:    
	PNG image data, 72 x 72, 8-bit/color RGBA, non-interl
\end{lstlisting}

\secup


\secrel{Документирование и CBL/M-learning}\label{docu}\secdown

\noindent
CBL\ --- Computer Based Learning, \term{автоматизированное обучение}

\begin{itemize}
\item документирование аппартно-программных разработок
\item передача знаний в рабочих группах
\item решение проблем описанных в \ref{codein}
\item пользовательская документация, справка
\end{itemize}
Документирование предполагает пассивный режим: есть мануалы и справка, но
пользователю они доступны в режиме чтения. При добавлении подсистемы CBL
появляется трекинг обучения конкретного пользователя
\begin{itemize}
  \item активное адаптивное обучение пользователей
  \item передача знаний в группах разработчиков и системных интеграторов
\end{itemize}

\secrel{M-learning: мобильное обучение}

\url{https://en.wikipedia.org/wiki/M-learning}

\bigskip
M-learning или \term{мобильное обучение}\ --- это «обучение в разных контекстах,
через социальные и контентные взаимодействия, с использованием персональных
электронных устройств. \emph{Форма дистанционного обучения}, 
использующая электронные образовательные технологии на мобильных устройствах
\emph{в удобное для обучающихся время}.

M-Learning ориентируется на мобильность учащегося, взаимодействующего с
портативными технологиями. Использование мобильных инструментов для создания
учебных пособий и материалов становится важной частью неформального обучения.

M-learning удобен тем, что доступен практически из любого места. Совместное
использование практически мгновенно среди всех, кто использует один и тот же
контент, что приводит к получению мгновенных отзывов и подсказок. Этот
высокоактивный процесс, как оказалось, улучшает результаты экзаменов с
50\% до 70\%, и сокрашает показатель отсева в
технических областях на 22\%. M-Learning также обеспечивает высокую
мобильность, заменяя книги и заметки небольшими устройствами, заполненными
специальным обучающим контентом. M-Learning имеет дополнительное преимущество,
заключающееся в том, что он эффективен с точки зрения затрат, поскольку цена
цифрового контента на планшетах резко падает по сравнению с традиционными
носителями (книги, CD и DVD и т.д.). Например, один цифровой учебник стоит от
одной трети до половины стоимости комплекта бумажных учебников, одновременно
предоставляя возможности: поиск, пользовательские ссылки, копирование учебных
материалов с произвольным редактированием.
 



\input{docu/modules}
\secrel{Метод Монте-Карло}\label{lrnkarlo}

\clearpage
Случайный показ произвольного объекта документации, включая узлы, которые на
него ссылаются, и узлы, на которые ведут исходящие ссылки. В результате
формируется \term{начальная точка просмотра} включающая взаимосвязи.
По каждому элементу пользователю предъявляется статистика покрытия, для
выявляения элементов документации, по которым пользователь прошел недостаточное
обучение.

\begin{itemize}
  \item 
Частота предъявления элементов выбирается в соответствии со статистикой
обучения, приоритет отдается областям \emph{смежным с наиболее посещаемыми} (они
интересуют пользователя так как статистика накручивается в процессе работы в
поисках ответов на проблемы).
  \item 
Одновременно в выборку \emph{добавляются области с
минимальным посещением, чтобы обеспечить равномерное покрытие} документации
просмотрами.
\end{itemize}


\secrel{Концепт-карта}

Концепт-карта (mindmap) представляет онтологию знаний в графическом
представлении (семантический граф). Нечеткая кластеризация позволяет выявить
элементы документации обладающие высокой взаимной корреляцией, и сгруппировать
их в 

\clearpage
\fig{docu/concmap.png}{height=\textheight}

\input{docu/scorm}

\input{docu/studtrack}

\secup



\addcontentsline{toc}{section}{Литература}
\begin{thebibliography}{99}

\clearpage
\bibitem{py}\ \bibfig{bib/python.jpg}\\
\textbf{Язык программирования Python}\\
Россум, Г., Дрейк, Ф.Л.Дж., Откидач, Д.С.,\ldots\\
\url{http://rus-linux.net/MyLDP/BOOKS/python.pdf}

\clearpage
\bibitem{dragon2} \bibfig{bib/dragon2.png}\ \emph{Purple Dragon Book /2nd ed/}\\
\textbf{Компиляторы: принципы, технологии и инструментарий} 2 изд.\\
Альфред В. Ахо, Моника С. Лам, Рави Сети, Джеффри Д. Ульман.\\
М.: Вильямс, 2008.\\ ISBN 978-5-8459-1349-4.\\
\url{https://www.ozon.ru/context/detail/id/148568229/}

\clearpage
\bibitem{sicp} \bibfig{bib/sicp.jpg}\ \textbf{\emph{SICP}\\
\href{https://drive.google.com/file/d/0B0u4WeMjO894X3lnWmhjUktKRk0/view?usp=sharing}{Структура
и интерпретация компьютерных программ}}\\
Харольд Абельсон, Джеральд Сассман\\
ISBN 5-98227-191-8\\
EN: \url{web.mit.edu/alexmv/6.037/sicp.pdf}\\
\url{https://www.ozon.ru/context/detail/id/5322055/}

\clearpage
\bibitem{bratko}\ \bibfig{bib/bratko.jpg}\\
\textbf{Программирование на языке Пролог\\для искусственного интеллекта}\\
Иван Братко\\
Мир, 1990\\ ISBN 5-03-001425-Х, 0-201-14224-4

\clearpage
\bibitem{minsky}\ \bibfig{bib/minsky.jpg}\\
\textbf{Фреймы для представления знаний}\\
Марвин Минский\\
 М.: Мир, 1979.\\
\url{https://royallib.com/book/minskiy_marvin/freymi_dlya_predstavleniya_znaniy.html}\\
\url{https://www.ozon.ru/context/detail/id/31747338/}

\clearpage
\bibitem{starting}\ \bibfig{bib/starting.jpg}\\
\textbf{Начальный курс программирования на языке ФОРТ}\\
Лео Броуди \\
М.: Финансы и статистика, 1990. - 352 с.  ISBN 5-279-00262-6.\\
Пер. с англ. В.А.Кондратенко. Под ред. Б.А.Кацева, В.А.Кириллина. Предисловие
И.В.Романовского.\\
\url{http://www.nncron.ru/download/sf.pdf}

\end{thebibliography}


\end{document}
