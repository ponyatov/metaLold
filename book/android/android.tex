\secrel{\A\ органайзер}\label{android}\secdown

\noindent
\begin{tabular}{l|p{8.3cm}}
\tfig{android/android_plan.png}{height=.45\textheight} &
Мобильниый телефон\ --- компьютер, который всегда с собой. Но \emph{удобных
средств программирования}, работающих на мобильном \textit{телефоне},
практически нет.
Разработка под \A\ и реализация on-device системы программирования\ --- тема
отдельной большой книги. Для начала можно попробовать написать \emph{органайзер,
программируемый пользоваталем}
\\ \end{tabular}

\medskip\noindent
На планшете есть несколько сред разработки для \emc, \py\ и \js, но на
телефоне они абсолютно неюзабельны.

\clearpage
\begin{description}
\item[PIM] Personal Information Manager

Функции, выполняемые органайзером\note{персональным информационным
менеджером}:
\begin{itemize}[nosep]
  \item 
\emph{планирование задач} для контроля за их самостоятельным и сторонним
выполнением (ToDo list, task-трекер, мобильный CRM);
  \item 
планирование событий, привязанных к датам и времени (праздники или встречи);
  \item 
\emph{напоминальники и зудильники} об определённых пользователем событиях;
  \item 
управление контактами (адресно-телефонная книга);
  \item 
записная книжка и листки-липучки;
  \item 
личные записи (дневник);
  \item 
интеграция с электронной почтой и мессенджерами
  \item 
\emph{персональная база знаний}.
\end{itemize}

\item[PPS] Personal Planning System, система персонального планирования\\
специально заточенная на трекинг задач, с точки зрения конкретного
человека.
\end{description}

\noindent
Несмотря на десятки лет усилий, даже такие гиганты как Google и Microsoft не
смогли решить проблему создания полноценного органайзера, в который by design
должен был превратиться смартфон:
\begin{itemize}
  \item доступны только примитивные типы задач, при этом на практике нужно
  множество вариантов, от простого будильника, встречи, действий привязанных по
  месту, периодические задачи с разным масштабом\note{от 30 минут для отдыха
  глаз, до года для дней рожденья}, до задач характерных для систем
  groupware
  \item полностью отсутствует функционал трекинга проектов: групповые
  задачи, зависимости задач по времени, исполнителям и ресурсам, делегирование и
  контроль,\ldots
  \item отсутствие средств индивидуальной адаптации, включая средства
  программирования пользователем, и доступ к внешним приложениям, библиотекам и
  сенсорам
\end{itemize}

\noindent
Для таких внутренне сложных приложений как органайзер, планировщик или трекер
задач, подход традиционных приложений с пользовательским GUI не подходит. Чем
больше автор приложения усложняет его, добавляя все новые и новые функции, тем
сложнее для пользователя становится его освоение. При этом даже для
очень переусложненного органайзера обязательно встретится случай использования,
под который не подходит ни один из предусмотренных в приложении вариантов.
Например потребуется задача, которая должна срабатывать одновременно по времени,
местоположению, и условию выполнения другой задачи, назначеная другому
пользователю в вашей рабочей группе.

\noindent
\begin{tabular}{l p{8.5cm}}
\tfig{android/basact.png}{height=.6\textheight} &
Для создания проекта в Android Studio лучше всего подходит Basic Activity, так
как из коробки в графическом интерфейсе приложения идет элемент добавления новой
задачи.

\medskip
\begin{tabular}{l l}
Name & HICO \\
Package name & \file{io.github.ponyatov.hico} \\
Save location & \file{/home/ponyatov/hico/Android} \\
Language & Java \\
Minimum API Level & API9 Android 2.3 \\
\end{tabular}
\\
\end{tabular}


\secup