\subsecly{Языково-ориентированное программирование}

Языково-ориентированное программирование\ --- разработка, опирающаяся на
предметно-специфичный язык (англ. DSL\ --- Domain-Specific Lan\-guage).
Это парадигма программирования, заключающаяся в разбиении процесса разработки
программного обеспечения на две стадии
\begin{itemize}[nosep]
  \item 
разработки предметно-ориентированных языков (DSL) и
  \item 
описания собственно решения задачи с их использованием.
\end{itemize}

\noindent
ЯОП и применение DSL-языков предназначено для контроля сложности разработки ПО
за счет разделения
\begin{description}
\item[машинно-зависимой части] и\\
множество тонкостей по выделению памяти, реализации
многопоточности, связи с внешними библиотеками и сервисами операционной системы,
библиотеки оптимизированных подпрограмм для численных расчетов, синтаксис
низкоуровневых языков, и т.п.
\item[проблемной части]\ \\
с которой взаимодествует (часто неподготовленный) пользователь рассматривающий
программный пакет с прикладной точки зрения: входной язык должен быть
максимально приближен к прикладной области решения задач, иметь хорошо читаемый
натуралистичный синтаксис, прикладной язык должен иметь возможность расширения
пользователем
\end{description}

\noindent
Такое разделение позволяет уменьшить экспоненциальный рост сложности, и
разделить сложности разработки, поддержки, и зону ответственности между
разработчиками платформы (низкий уровень) и прикладными программистами и
интеграторами (проблемный уровень). Типичный пример\ --- система 1С.

Использование DSL вместо языков общего назначения существенно повышает уровень
абстрактности кода, что позволяет вести разработку быстро и эффективно и
создавать программы, которые легки в понимании и сопровождении; а также делает
возможным или существенно упрощает решение многих задач, связанных с
метапрограммированием (порождение программ, проверка корректности кода,
трансформации).

ЯОП выделяется гораздо более агрессивной направленностью на приближение
компьютера к человеку. Среди последователей ЯОП бытует мнение, что в
ресурсоёмких задачах \emph{хорошо спроектированный и реализованный DSL} делает
общение неподготовленного\note{с точки зрения навыков программиста и специалиста
по автоматизации ПО} человека с компьютером \emph{куда более удобным и
продуктивным, чем графический интерфейс пользователя}.
