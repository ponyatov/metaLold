% \clearpage
\subsecly{Гомоиконичные языки программирования}\label{homoiconic}

\begin{quotation}\noindent
\term{Гомоиконичность} (гомоиконносль, англ. homoiconicity, homoiconic)\\
свойство некоторых языков программирования, в которых \emph{представление
программ является одновременно структурами данных} определенных в типах самого
языка, \emph{доступных для просмотра и модификации}. Говоря иначе,
гомоиконичность\ --- это когда исходный \textit{код программы} пишется
\textit{как базовая структура данных}, и язык программирования знает, как
получить к ней доступ (в том числе в рантайме).
\end{quotation}

\noindent
В гомоиконичном языке \textit{метапрограммирование это основная методика}
разработки ПО, использующаяся в том числе и \textit{для расширения языка} до
возможностей, нужных конкретному программисту.

В качестве первого примера всегда приводится язык \lisp, который был создан для
обработки данных, представленных в форме вложенных списков.
Лисп-программы тоже записываются, хранятся и выполняются в виде списков; в
результате получается, что программа во время работы может получить доступ к
своему собственному исходному кода, а также автоматически изменять себя «на
лету». Гомоиконичные языки, как правило, включают полную поддержку
\term{синтаксических макросов}, позволяющие программисту определять новые
синтаксические структуры, и выражать \term{преобразования программ} в компактной
форме.
