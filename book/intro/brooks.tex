\subsecly{Серебряная пуля Брукса}

Сущностью программирования является, прежде всего, не написание инструкций на
конкретном языке программирования, а выработка подробной структуры
взаимодействующих сущностей проблемной области, а также проверка внутренней
непротиворечивости этой структуры.
Следовательно, ни одно средство разработки ПО не сможет существенно снизить
сложность разработки, так как даже если, например, изобрести компьютерный язык,
оперирующий понятиями на уровне проблемной области, программирование все равно
останется сложной задачей, поскольку придется точно определять взаимосвязи между
объектами реального мира, устанавливать исключения, предусматривать все
возможные переходы между состояниями и т.д.

Что делает язык высокого уровня? Они изолируют программу от большей части ее
нецелевой сложности. Абстрактная программа состоит из концептуальных
конструкций: операций, типов данных, последовательностей и коммуникации.
Конкретная машинная программа связана с битами, регистрами, условиями, ветвями,
каналами, дисками и тому подобным. В той степени, в которой язык высокого уровня
воплощает конструкции, которые нужны в абстрактной программе, и избегает всех
нижестоящих, он устраняет целый уровень сложности, который вообще никогда не был
присущ программе. Безусловно, уровень нашего мышления о структурах данных, типах
и операциях неуклонно растет в сторону прикладной области благодаря
возможностям абстракции, предоставляемым ООП, но с постоянно уменьшающейся
скоростью\ --- развитие языков и абстракий движется все ближе и ближе к
прикладной сложности пользователей. Более того, в какой-то момент разработка
языка высокого уровня и фреймворков создает бремя владения инструментом, которое
увеличивает, а не уменьшает интеллектуальную сложность пользователя.

Многие люди ожидают, что достижения в области искусственного интеллекта
обеспечат революционный прорыв, который даст увеличение производительности и
качества программного обеспечения на порядок. При этом не стоит путать
"численный ИИ" как он широко известен сейчас, т.е. нейроные сети и машинное
обучение, с \term{семантическим ИИ}, выполняющим логический вывод \emph{на
основе сетей взаимосвязанных понятий и отношений между объектами}. Наиболее
широко известная технология семантического ИИ\ --- \term{экспертные системы}.

\begin{quotation}\noindent
Экспертная система\ --- это программа, которая содержит обобщенный механизм
логического вывода и базу правил и отношений, принимает входные данные и
предположения, генерурет гипотезы, выводимые из базы правил, дает выводы и
рекомендации, и предлагает объяснение свои результатов для пользователя (путем
отслеживания цепочки логического вывода). Механизмы вывода часто могут иметь
дело с нечеткими или вероятностными данными и правилами, в дополнение к чисто
детерминированной логике.
\end{quotation}

Как эта технология может быть применена к задаче разработки программного
обеспечения?

Работа, необходимая для генерации базы знаний\ --- это работа, которую в любом
случае необходимо будет не просто выполнить единожды, но и постоянно
поддерживать базу знаний в актуальном состоянии. Многие трудности стоят на пути
скорейшей реализации полезных экспертных системных советников для разработчика
программ. Важной частью нашего воображаемого сценария является разработка
простых способов перехода от спецификации структуры программы к автоматической
или полуавтоматической генерации кода, созданию правил тестирования и
диагностики, скриптов развертывания и средств мониторинга. Еще более трудной и
важной проблемой является получение знаний в двух направлениях: поиск четких,
самоаналитических экспертов, которые знают, почему они делают что-то, и
разработка эффективных методов извлечения того, что они знают, и их
использование в базах правил. Необходимым условием для построения экспертной
системы является наличие эксперта.

Самым мощным вкладом экспертных систем, безусловно, должно быть предоставление
на службу неопытному программисту опыта и накопленной мудрости лучших
программистов. Это немалый вклад. Разрыв между лучшей практикой разработки
программного обеспечения и средней практикой очень велик\ --- возможно, больше,
чем в любой другой инженерной дисциплине. Инструмент, который распространяет
передовой опыт, был бы важен.

