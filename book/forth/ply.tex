\secrel{Парсер с использованием библиотеки PLY}\label{ply}

Библиотека \file{PLY} \ref{ply}\ позволяет писать \term{парсеры} для достаточно
сложных языков. Несмотря на то что диалекты \F/\pyf\ требуют только реализацию
\term{лексера}, есть смысл немного сэкономить усилия, и не заморачиваться с
написанием традиционного посимвольного разбора ``до пробела''. Если вам для
какой-то задачи потребуется применение \term{инфиксного синтаксиса}, типа
разбора арифметических выражений, вы сможете без особых усилий добавить для них
\term{синтаксический анализатор}.

Еще одно достоинство использования PLY: с ее помощью мы можем автоматически
определять тип для каждой лексемы, в частности разпознавать примитивные типы
\ref{prim}\ и вызывать соответствующие конструкторы. В \ref{typeval}\ есть
особая сноска по этому поводу: для совместимости с PLY были специально выбраны
имена переменных типа и значения: \verb|type| и \verb|value|.

Подробное использование библиотеки \file{PLY} см. \ref{ply}.

\clearpage
\lst{forth/parser.py}{language=Python}

\clearpage
\begin{description}%[nosep]
\item[tokens] список токенов, которые может распознавать парсер\\
так как мы специально обеспечили возможность использования фрей\-мов-примитивов
в качестве \term{литералов}, в этом списке должны быть перечислены тэги (c
маленькой буквы)
\item[t\_ignore] символы которые не будут участвовать в разборе (пробелы) 
\item[t\_number()] правило лексера распознающее числа
\item[t\_symbol()] правило распознающее символы как имена форт-слов
\item[t\_error()] обработка синтаксических ошибок: нераспознанные символы
\end{description}

% \medskip
\lst{forth/lexer.py}{language=Python}
